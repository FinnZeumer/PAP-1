\chapter{Diskussion}
\label{sc:dis}

\section{Zusammenfassung}
In der ersten Aufgabe wurde die Faraday-Konstante mit Kupferelektroden bestimmt. 
Die Massenänderung an Anode und Kathode ergab für die Konstanten
\begin{align}
    F_{Ka} = (93\,000 \pm 4\,000)\,\frac{C}{mol}, \\ 
    F_{An} = (96\,000 \pm 4\,000)\,\frac{C}{mol}.    
\end{align}

Beide Werte stimmen innerhalb der Unsicherheiten miteinander und mit dem Literaturwert $F_{lit} = 96\,485\,\frac{C}{mol}$ überein.  
In der zweiten Aufgabe wurde die Elektrolyse von Wasser untersucht. Aus den Volumina der Gase ergaben sich 
\begin{align}
    F_{H_2} = (97\,000 \pm 10\,000)\,\frac{C}{mol}, \\
    F_{O_2} = (87\,000 \pm 18\,000)\,\frac{C}{mol}.
\end{align}

Auch diese Werte sind konsistent mit dem Literaturwert, wenn auch mit deutlich größeren Unsicherheiten.

\begin{table}[h!]
\centering
\begin{tabular}{c|c|c|c}
\toprule
 & $F_{lit}$ & $F_{H_2}$ & $F_{O_2}$ \\
\midrule
$F_{lit}$ & $0,00\sigma$ & $0,05\sigma$ & $0,53\sigma$ \\
$F_{An}$  & $0,12\sigma$ & $0,09\sigma$ & $0,49\sigma$ \\
$F_{Ka}$  & $0,87\sigma$ & $0,37\sigma$ & $0,33\sigma$ \\
\bottomrule
\end{tabular}
\caption{Signifikanz der Abweichungen zwischen den bestimmten Faraday-Konstanten und dem Literaturwert.}
\label{tab:sigmas}
\end{table}


\section{Diskussion}
Die Messungen mit den Kupferelektroden lieferten präzisere Ergebnisse. 
Dies liegt an den klar bestimmbaren Massendifferenzen und der guten Auflösung der Waage. 
Die Abweichung zwischen Anode und Kathode betrug nur $0,53\sigma$, was eine hohe interne Konsistenz bestätigt.  
Die Genauigkeit wird hauptsächlich durch die Unsicherheit des Stromes bestimmt, während die Zeitmessung und Waagenungenauigkeit vernachlässigbar klein sind. 
Insgesamt konnte hier die Faraday-Konstante auf wenige Prozent genau bestimmt werden.

Die Bestimmung über die Wasserstoff- und Sauerstoffvolumina war wesentlich ungenauer. 
Die Ursache liegt in der Schätzung der Elektrolysedauer, der Ablesegenauigkeit der Volumina und den notwendigen Korrekturen für Wasserdampfdruck und Temperatur. 
Zusätzlich reagiert ein Teil des Sauerstoffs mit der Schwefelsäure, was systematische Fehler verursacht. 
Die dadurch resultierenden Unsicherheiten sind deutlich größer, führen jedoch immer noch zu Ergebnissen, die mit dem Literaturwert vereinbar sind.  

Bemerkenswert ist, dass die Wasserstoffmessung eine kleinere relative Unsicherheit aufweist als die Sauerstoffmessung. 
Dies ist durch das doppelte Gasvolumen erklärbar, welches den Einfluss der Ablesefehler reduziert. 
Die Sauerstoffwerte sind daher weniger zuverlässig und zeigen eine stärkere Abweichung, die aber statistisch nicht signifikant ist.  
Zudem reagiert der Sauerstoff mir der Schwefelsäure, wodurch es zu Ungenauigkeiten kommt.

\section{Kritik}
Für die Kupferelektroden ist die Annahme einer konstanten Stromstärke nur näherungsweise korrekt. 
Schwankungen des Netzteils oder Kontaktwiderstände wurden nicht erfasst. 
Weiterhin können kleine Mengen an Kupferionen im Elektrolyten komplexiert oder abgeschieden werden, was die Massenbilanz leicht verzerrt.  
Zuletzt sind unumgänglich Cu-Prtikel >>verloren<< gegangen, da sich einige in dem Elektrolyt-Gemisch befanden, als der Strom abgestellt wurde, daher hat die Anode mehr Masse verloren, als die Kathode aufgenommen hat.

Bei der Wasser-Elektrolyse stellt die ungenaue Zeitmessung die größte Unsicherheit dar. 
Die Annahme einer konstanten Stromstärke über die gesamte Zeit kann nicht garantiert werden. 
Die Volumenablesung hängt stark vom Auge des Beobachters ab und ist fehleranfällig, insbesondere durch Parallaxen. 
Zusätzlich wurde der Einfluss von Blasenbildung und der unvollständigen Gastrennung nicht berücksichtigt.  
Ein weiterer realistischer Fehler ist die mögliche Gaslösung im Wasser, wodurch gemessene Volumina geringer erscheinen.  
