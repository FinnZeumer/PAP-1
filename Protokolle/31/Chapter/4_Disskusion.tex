\chapter{Diskussion}
\label{ch:diskussion}

\section{Zusammenfassung}

Im Folgenden sind die wichtigsten experimentellen Ergebnise zusammengefasst und kurz erläutert. 

Zunächst wurde die Brennweite einer achromatischen Linse bestimmt. Die Meßpunkte zeigten dabei eine deutliche Streuung, was auf erhebliche Unsicherheiten bei der Messung der Gegenstands- und Bildweiten hindeutet. Eine präzise Brennweitenbestimmung war daher nihct möglich.

Für die bikonvexe Linse wurde die Brennweite mit dem Bessel-Verfahren bestimmt. Das Ergebnis lautet:
\begin{equation}
    f = (12{,}13 \pm 0{,}05)\,\mathrm{cm}
\end{equation}
Dieser Wert stimmt sehr gut mit der Herstellerangabe von $f_{\text{lit}} = 12\,\mathrm{cm}$ überein und bestätigt die hohe Genauigkeit des Bessel-Verfahrens.

Zur Untersuchung des Einflusses verschiedener Blenden auf die Brennweite wurden sowohl eine Loch- als auch eine Ringblende verwendet. Die gemessenen Brennweiten betragen:
\begin{align}
    f_{\text{Lochblende}} &= (12{,}437 \pm 0{,}028)\,\mathrm{cm} \\
    f_{\text{Ringblende}} &= (12{,}140 \pm 0{,}028)\,\mathrm{cm}
\end{align}
Die Lochblende ließ vorwiegend achsennahe Strahlen passieren, was zu einer leicht größeren Brennweite führte. Die Ringblende dagegen ließ nur Randstrahlen durch, wodurch die Brennweite geringer ausfiel. Beide Werte liegen jedoch sehr nah beieinander, was eine insgesamt gute optische Qualität der Linse bestätigt.

Die chromatische Aberration wurde mithilfe farbiger Lichtfilter untersucht. Dabei ergaben sich für rotes und blaues Licht folgende Brennweiten:
\begin{align}
    f_{\text{rot}} &= (11{,}97 \pm 0{,}08)\,\mathrm{cm} \\
    f_{\text{blau}} &= (11{,}81 \pm 0{,}10)\,\mathrm{cm}
\end{align}
Das Ergebnis zeigt, dass die Brennweite für rotes Licht geringfügig größer ist als für blaues Licht. Dies entspricht der physikalischen Erwartung, da Licht längerer Wellenlänge (rot) schwächer gebrochen wird als Licht kürzerer Wellenlänge (blau).

Im abschließenden Versuchsteil wurde das Auflösungsvermögen eines Mikroskops untersucht. Dabei wurden folgende Werte bestimmt:
\begin{align}
    b_G &= (0{,}500 \pm 0{,}007)\,\mathrm{mm} \\
    G   &= (0{,}095 \pm 0{,}003)\,\mathrm{mm} \\
    G_{\text{min}} &= (790 \pm 70)\,\mathrm{nm}
\end{align}
Während die gemessene Gitterkonstante $G$ plausibel erscheint, liegt der berechnete minimale auflösbare Gitterabstand $G_{\text{min}}$ um etwa drei Größenordnungen darunter. Dies weist vermutlich auf einen Einheitenfehler oder eine falsche Skalierung bei der Auswertung hin. Trotz dieser Abweichung bestätigen die Messergebnisse Grundsätzlich die theoretisch erwarteten Zusammenhänge zwischen Wellenlänge, Auflösungsvermögen und optischer Geometrie.


Die Ergebnisse der zweiten und dritten Aufgaben zeigen eine hohe Präzision und gute Übereinstimmung mit theoretischen Erwartungen, während die Messreihe zur achromatischen Linse deutliche Abweichungen aufweist.

\section{Diskussion}

In der ersten Aufgabe konnte die Brennweite der achromatischen Lins nur ungenau bestimmt werden. Die Punkte in der Auftragung von Bildweite gegen Gegenstandsweite zeigten keine klare Schnittstelle, sodass sich keine eindeutige Brennweite aus dem Diagramm ablesen ließ. Besonders auffällig war die starke Streuung der Messpunkte, was auf erhebliche Messunsicherheiten sowohl bei der Bestimung von $g$ als auch von $b$ hindeutet. Ein Vergleich der Abweichungen zeigt, dass die Gegenstandsweiten im Mittel präziser bestimmt wurden, während die Bildweiten deutlich größere Unsicherheiten aufwisen. Dies lässt vermuten, dass der Hauptfehler in der Scharfstellung des Bildes lag – insbesondere, weil das Bild bei kleinen Gegenstandsweiten häufig unscharf war und der Fokuspunkt nur schwer festzulegen war.

Die Ergebnisse des Bessel-Verfahrens hingegen waren deutlich konsistenter. Die gemessene Brennweite von $f = (12{,}13 \pm 0{,}05)\,\mathrm{cm}$ stimmt hervorragend mit dem Literaturwert überein. Hier zeigt sich, dass das Bessel-Verfahren eine robuste und relativ fehleruenmpfindliche Methode zur Brennweitenbestimmung darstellt. Auch die Untersuchung der Aberrationen führte zu plausiblen Ergebnissen. Der Einfluss der Loch- und Ringblende war wie erwartet: Die Lochblende ließ nur achsennahe Strahlen durch und führte zu einer geringfügig größeren Brennweite, während die Ringblende nur Randstrahlen durchließ und dadurch eine geringfügig kleinere Brennweite ergab. Die Unterschiede waren zwar klein, aber statistisch signifiknt.

Bei der chromatischen Aberration zeigte sich, dass rotes Licht eine etwas größere Brennweite als blaues Licht aufweist. Dies ist Physikalisch konsistent, da Licht mit größerer Wellenlange (rot) weniger stark gebrochen wird. Die Abweichung ist zwar gering, aber dennoch signifikant. Damit wurde das Prinzip der Dispersion durch die Linse experimentell bestätigt.

Die mikroskopische Untersuchung des Gitters führte allerdings zu widersprüchlichen Ergebnissen. Die berechnete Gitterkonstante $G = (0{,}095 \pm 0{,}003)\,\mathrm{mm}$ weicht um mehr als 30 Standardabweichungen vom berechneten minimal auflösbaren Gitterabstand $G_{min} = (790 \pm 70)\,\mathrm{nm}$ ab. Eine so große differenz deutet klar auf einen gravierenden Fehler in der Versuchsdurchführung oder Datenaufzeichnung hin. Wahrscheinlich wurde ein Wert in falscher Einheit oder Größenordnung notiert (zum Beispiel ein Wert in Zentimetern statt Millimetern). Da die Größenordnung um etwa drei Zehnerpotenzen abweicht, ist ein solcher Notationsfehler oder eine Verwechslung zwischen Maßstäben (z. B. Mikrometer und Millimeter) am wahrscheinlichsten.

\section{Kritik}

Die Hauptkritik am Versuch liegt in der mangelnden Sorgfalt bei der Datenerfassung, insbesondere in der ersten und letzten Messreihe. Die unpräzise Erfassung der Bildweiten in der ersten Aufgabe führte dazu, dass keine konsistente Brennweitenbestimmung möglich war. Messwerte sollten zukünftig mehrfach kontrolliert und auf Plausibilität überprüft werden, insbesondere wenn sich Werte stark voneinander unterscheiden.

Auch im Mikroskop-Versuch deutet die extreme Abweichung zwischen $G$ und $G_{min}$ auf unsauberes Arbeiten hin. Hier wäre eine gründlichere Dokumentation der Einheiten und Messbedingungen erforderlich gewesen, um eine Fehl Interpretation auszuschließen.

Zukünftige Versuche sollten daher mit besonderem Augenmerk auf sorgfältiges, systematisches Messen und konsequente Überprüfung der Einheiten durchgeführt werden, um solche Inkonsistenzen zu vermeiden. Trotz der teilweise ungenauen Ergebnisse lassen sich die grundlegenden physikalischen Zusammenhänge aus den Messreihen jedoch klar erkennen.
