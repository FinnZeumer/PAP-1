\chapter{Diskussion}

\section{Zusammenfassung}  
In diesem Versuch wurden die spezifischen Wärmekapazitäten von Graphit, Aluminium und Blei bei Raumtemperatur und bei tiefen Temperaturen (Flüssigstickstoff) bestimmt. Dazu wurden zwei Messmethoden verwendet: ein Wasserkalorimeter zur Messung im Temperaturbereich \(\SI{20}{\celsius} - \SI{100}{\celsius}\) und ein Dewargefäß mit flüssigem Stickstoff zur Bestimmung bei \(\SI{-196}{\celsius}\).  

Die gemessenen Werte zeigen eine starke Streuung, insbesondere bei Graphit und Aluminium. Die experimentell ermittelten Molwärmen weichen zum Teil erheblich von den Literaturwerten und vom Dulong-Petit-Wert ab. Für Blei ergaben sich Werte, die zwar näher an der Literatur liegen, jedoch aufgrund der Messunsicherheit ebenfalls nur bedingt zuverlässig sind.  

Die Debye-Temperaturen, abgeleitet aus den Niedrigtemperatur-Messungen, liegen weit unterhalb der Literaturwerte für Graphit und Aluminium, während sie für Blei zumindest grob mit den erwarteten Größenordnungen übereinstimmen.

\section{Diskussion}  
Die Ergebnisse zeigen, dass die experimentelle Genauigkeit stark von der Methodik abhängig ist. Die Messungen mit dem Wasserkalorimeter liefern teilweise unrealistische Ergebnisse (z.B. sehr große Fehler bei Graphit), was auf systematische Probleme bei der Temperatur- und Massemessung hindeutet.  

Die Messungen im Flüssigstickstoff sind vergleichsweise genauer, liefern jedoch für Graphit und Aluminium weiterhin zu niedrige Molwärmen. Dies ist physikalisch plausibel, da das Gesetz von Dulong-Petit bei tiefen Temperaturen nicht gilt: Die spezifische Wärme von Festkörpern fällt gegen \(T \to 0\) ab, wie das Debye-Modell beschreibt. Für Blei hingegen liegen die Werte näher am erwarteten Bereich, was an seiner geringen Debye-Temperatur liegt.  

Die starke Diskrepanz zwischen Messwerten und Literaturwerten für Graphit (Molwärme bei Raumtemperatur stark unterschätzt) deutet außerdem auf systematische Messfehler hin. Dazu zählen z.B. Wärmeverluste an die Umgebung, ungenaue Temperaturablesung oder Unsicherheiten bei der Bestimmung der Masse des verdampften Stickstoffs.
Hier hat es auch keinen nenneswerten Unterschied gemacht, ob man die leichten oder schweren Massen in den Flüssigenstickstof hält und auswertet; die Werte bleiben entweder statistisch signifikant, oder weichen weiterhin stark ab.
Das die Bleiwerte überhaupt statistisch signifikant sind, kann man ziemlich eindeutig auf die Ungenauigkeit zurückführen, da die Werte immer die gleiche ungenauigkeit haben, wie der Wert selbst. 
Dennoch bleibt der Wert am statistisch Signifikanten; für den Leichten Körper bei einer Ungenauigkeit von $\pm 50K$, was $50\%$ der Skaleneinheit entspricht,
käme man so auf eine $2,1\sigma$-Abweichung; somit wäre der Wert statistisch eher nicht signifikant. Jedoch spannend ist die schwere Masse, bei der wir eine $0,6\sigma$-Abweichung haben, dies wäre nochimmer statistisch signifikant.
Dennoch war die bestimmungsmethoede der Bleiwerte geraten, und nicht wissenschaftlich bestimmt, jegliche Aussagekraft ist damit sowieso hinüber.

\section{Kritik}  
Die experimentellen Unsicherheiten sind teilweise sehr hoch, was die Aussagekraft der Ergebnisse stark einschränkt. Kritische Punkte sind:  

\begin{enumerate}
    \item \textbf{Messmethodik:} 
    \begin{itemize}
        \item Beim Wasserkalorimeter können Wärmeverluste an die Umgebung nicht vernachlässigt werden.  
        \item Die Temperaturmessung ist nur auf \(\pm 0{,}3\%\) genau, was insbesondere bei kleinen Temperaturdifferenzen den Fehler überproportional vergrößert.  
    \end{itemize}
    
    \item \textbf{Massenbestimmung:}  
    Besonders beim Flüssigstickstoff ist das Wiegen des Dewargefäßes kritisch, da kleinste Differenzen die berechnete Wärmekapazität stark beeinflussen.  

    \item \textbf{Probenfehler:}  
    Unterschiedliche Größen und Geometrien der Testkörper führen zu abweichenden Wärmeverlusten, da die Oberfläche wichtig für den Wärmeaustausch ist.
    
    \item \textbf{Methodische Einschränkungen:}  
    \begin{itemize}
        \item Die Annahme, dass die gesamte Wärme nur vom festen Körper an den Flüssigstickstoff abgegeben wird, ist idealisiert.  
        \item Die graphische Bestimmung der Debye-Temperatur ist stark subjektiv und mit großen Unsicherheiten behaftet (z.B. bei Blei nur als Mittelwert zwischen 0K und Maximalwert geschätzt).  
    \end{itemize}
\end{enumerate}

Insgesamt zeigt der Versuch anschaulich, dass die Bestimmung spezifischer Wärmekapazitäten mit einfachen Methoden grundsätzlich möglich ist, die erzielten Werte aber stark durch systematische und statistische Fehler verfälscht werden. Einige Ergebnisse stimmen eher zufällig gut mit Literaturwerten überein, ohne dass dies garantiert auf die Genauigkeit der Messung zurückzuführen ist.
