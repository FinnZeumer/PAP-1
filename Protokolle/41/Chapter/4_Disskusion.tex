\chapter{Diskussion}

\section{Zusammenfassung, Diskussion und Kritik}

\subsubsection*{Eichung Gasthermometer}
In der ersten Aufgabe wurde eine Eichkurve des Gasthermometers erstellt. Dafür wurden die Eichpunkte bei $0 \, ^\circ C$ und bei:
\begin{equation}
    E_{100^\circ C} = (1254,7 \pm 0,6) \, hPa
\end{equation}
nach \hyperref[e:1]{Gleichung 3.14} verwendet.

Über diese Kurve ergaben sich die Werte für \hyperref[e:3]{Flüssigstickstoff und Trockeneis (3.16/3.17)} sowie für den \hyperref[e:2]{absoluten Nullpunkt (3.15)}:
\begin{align}
    T_0 &= (-275,8 \pm 13)  \, ^\circ C  \\
    T_{N_2} &= (-197,16 \pm 10,4) \, ^\circ C \\
    T_{TE} &= (-58,346 \pm 5,2) \, ^\circ C
\end{align}

Der Wert für den Flüssigstickstoff liegt in der $0,13\sigma$-Umgebung des Literaturwerts. Mit der »verbesserten« Eichkurve ergab sich eine Abweichung von 
$0,16\sigma$, also ein ähnlicher, aber etwas größerer Wert. Dies hängt vermutlich mit der geringeren Unsicherheit des zweiten Werts bei $T_{0,2} = (-271,68 \pm 9,1) \, ^\circ C$ zusammen.

Der Wert für das Trockeneis weicht hingegen um $5,8\sigma$ ab und ist damit nicht statistisch signifikant.

\subsection*{PT100-Thermometer}
Auch in dieser Aufgabe wurde ein Thermometer geeicht, diesmal ein Widerstandsthermometer. Der graphisch bestimmte Wert wurde mit dem rechnerisch angenäherten Wert verglichen. Die Abweichung betrug $1,79\sigma$. Damit liegt zwar eine Abweichung vor, diese ist aber noch im erwartbaren Bereich.
Zu beachten ist, dass die Rechnung nur eine Annäherung darstellt und graphische Bestimmungen grundsätzlich ungenau sind. Eine etwas größere Abweichung war daher zu erwarten. 
Als Fehlerquellen kommen vor allem ungenaue Graphen, begrenzte Messgeräte, wenige Messreihen und Ableseungenauigkeiten infrage.

\subsection*{Gasthermometer vs. Pyrometer}
In diesem Aufgabenteil sollte das Pyrometer im Vergleich zum geeichten Gasthermometer bewertet werden. Das Ergebnis lautete:
\begin{equation}
(91,76 \pm 3,77)\%.
\end{equation}

Das Pyrometer zeigte also im Durchschnitt etwa den 0,92-fachen Wert des Gasthermometers an. Für die meisten Anwendungen wäre dies unzureichend, da Werte im Bereich von 98\% bis 99,9\% wünschenswert wären. Ein Teil der Abweichung dürfte bereits durch die Eichung des Gasthermometers entstehen. Zudem weist das Pyrometer eine eigene Messunsicherheit auf. 
Insgesamt ist der Vergleich auf einer graphischen Auswertung basiert, die grundsätzlich mit Unsicherheiten behaftet ist. 

\subsection*{Flammenanalyse}
Im letzten Aufgabenteil wurde die Flammentemperatur an fünf verschiedenen Positionen bei schwacher und starker Luftzufuhr bestimmt.
Die Ergebnisse weichen jedoch stark von den Literaturwerten ab, die für die meisten Bunsenbrenner angegeben sind \cite{wikipedia-bunsenbrenner}:
\begin{equation}
    T_{Brenner} = \{350;1300\} \, ^\circ C.
\end{equation}

Die gemessenen Werte lagen lediglich im Bereich von:
\begin{equation}
    T_{Gem} = \{18;121\} \, ^\circ C.
\end{equation}

Dies entspricht Abweichungen von:
\begin{align}
    \frac{\left| 350-18 \right|}{3,54} = 93,79\sigma \text{ und }\\
    \frac{\left| 1300-133,58 \right|}{3,16} = 369,12\sigma.
\end{align}

Eine detaillierte Auswertung dieser Abweichungen ist nicht sinnvoll, da die Ergebnisse offensichtlich nicht konsistent sind. 
Vermutlich trat bei der Messung ein Fehler auf. Wenn die gemessenen Spannungen um den Faktor 10 vergrößert werden, ergeben sich folgende Werte:

\begin{table}[h!]
    \onecolumn
    \centering
    \caption{Korrigierte Werte (mit Faktor 10)}
    \label{tab:letzte_tabelle}
    \begin{tabular}{c | c | c | c | c | c | c}
        \toprule
        Luftzufuhr & Position & $U [mV]$ & $\Delta U [mV]$ & $T_{TE} [^\circ C]$ & $\Delta_{min} T [^\circ C]$ & $\Delta T [^\circ C]$ \\
        \midrule
        Schwach & 1 & 1,0 & 0,025 & 146,31 & 143,15 & 3,16 \\
                & 2 & 3,0 & 0,035 & 372,75 & 369,05 & 3,70 \\
                & 3 & 4,5 & 0,0425 & 526,82 & 522,51 & 4,31 \\
                & 4 & 7,0 & 0,055 & 768,01 & 762,98 & 5,03 \\
                & 5 & 8,0 & 0,06 & 859,69 & 854,27 & 5,42 \\
        \hline
        Stark   & 1 & 8,0 & 0,06 & 859,69 & 854,27 & 5,42 \\
                & 2 & 8,0 & 0,06 & 859,69 & 854,27 & 5,42 \\
                & 3 & 8,0 & 0,06 & 859,69 & 854,27 & 5,42 \\
                & 4 & 8,0 & 0,06 & 859,69 & 854,27 & 5,42 \\
                & 5 & 9,0 & 0,065 & 948,77 & 943,05 & 5,72 \\
        \bottomrule
    \end{tabular}
    \twocolumn
\end{table}

Die Werte sind zwar noch relativ niedrig, und bei starker Luftzufuhr tritt kaum Variation auf, was eigentlich zu erwarten wäre. 

Die Abweichungen fallen jedoch bereits deutlich geringer aus:
\begin{align}
    \frac{\left| 350-146,31 \right|}{3,16} = 65,40\sigma \text{ und }\\
    \frac{\left| 1300-948,77 \right|}{5,72} = 61,40\sigma.
\end{align}

Die Ergebnisse sind zwar weiterhin unzureichend, könnten jedoch plausibel sein, wenn der verwendete Bunsenbrenner tatsächlich im Bereich von 
150 bis 1000 Grad Celsius arbeitet.
