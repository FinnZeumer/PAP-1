\chapter{Durchführung}

\section{Versuchsaufbau}

\subsection*{Aufgabe 1: Eichung der Thermometer bei 0\,°C}
Für die Eichung wird ein Pt100-Thermometer mit Adapterbox verwendet. Die vier Anschlussleitungen können über 4\,mm-Buchsen abgegriffen werden. Eine Stromquelle liefert den Messstrom von 1\,mA. Die Spannungsmessung erfolgt mit einem Voltmeter, das entweder an die Buchsen der Stromquelle (Zweileiterschaltung) oder direkt an die Adapterbox (Vierleiterschaltung) angeschlossen wird.  
Zur Erzeugung der Referenztemperatur von 0\,°C wird ein Becherglas mit zerkleinertem Eis und Wasser befüllt. Der Glasballon mit dem Messsensor wird vollständig in das Gemisch eingetaucht. Zusätzlich stehen ein Pyrometer zur Messung der Oberflächentemperatur sowie ein Flüssigkeitsthermometer als Vergleich zur Verfügung.

\subsection*{Aufgabe 2: Temperaturmessung bis 100\,°C}
Das Wasserbad wird mit einer Heizplatte erhitzt. Zur Homogenisierung der Temperatur kommt ein Rührmechanismus zum Einsatz. Als Messgeräte dienen das Pt100-Thermometer, das Gasthermometer und ein Pyrometer. Der Umgebungsdruck wird mit einem Barometer erfasst.

\subsection*{Aufgabe 3: Temperaturmessung bei tiefen Temperaturen}
Ein Dewargefäß wird wahlweise mit einer Trockeneis-Alkohol-Mischung oder mit flüssigem Stickstoff gefüllt. Der Glasballon mit dem Messsensor wird in das jeweilige Kühlmedium eingebracht. Die Messung erfolgt mit dem Pt100-Thermometer sowie dem Gasthermometer.

\subsection*{Aufgabe 4: Temperaturmessung mit dem PtRh-Thermoelement}
Zur Untersuchung hoher Temperaturen wird ein Gasbrenner mit regelbarer Luftzufuhr genutzt. Das PtRh-Thermoelement (Typ S oder Typ B) wird in verschiedene Bereiche der Flamme eingeführt. Die Temperaturbestimmung erfolgt über die gemessene Thermospannung unter Zuhilfenahme der passenden Eichtabelle.

\section{Messverfahren}

Nach dem Aufbau der Zwei- und Vierleiterschaltung werden beide Varianten getestet. Bei stabilisierter Temperatur im Wasser-Eis-Gemisch werden die Spannung des Pt100, der Druck des Gasthermometers sowie die Temperatur des Pyrometers aufgezeichnet. Das Flüssigkeitsthermometer dient als zusätzliche Kontrolle. Maßgeblich ist das Minimum der Pt100-Spannung als Eichpunkt.

Anschließend wird das Wasser stufenweise erhitzt. Beginnend bei etwa 10\,°C werden in Schritten von ca. 10\,°C die Pt100-Spannung, der Gasthermometerdruck und die Pyrometertemperatur aufgenommen. Als letzter Messpunkt dient die Temperatur des siedenden Wassers. Der Luftdruck wird parallel am Barometer erfasst. Das Pyrometer wird schräg auf die Wasseroberfläche gerichtet, um Verfälschungen durch Wasserdampf zu vermeiden.

Für tiefe Temperaturen wird zunächst die Trockeneis-Alkohol-Mischung eingesetzt. Nach ausreichender Abkühlung und Temperaturstabilisierung werden Pt100-Spannung und Druck gemessen. Anschließend wird der Versuch mit flüssigem Stickstoff wiederholt. Der Glasballon wird vollständig eingetaucht, und die Messwerte werden nach Abklingen der starken Verdampfung notiert. Das Pyrometer wird hier nicht verwendet, da es bei diesen Temperaturen ungeeignet ist.

\subsection*{Aufgabe 4: Temperaturmessung mit dem PtRh-Thermoelement}
Das Thermoelement wird in die Flamme des Gasbrenners eingeführt. Bei schwacher und starker Luftzufuhr werden jeweils mehrere Messpunkte in unterschiedlichen Flammenzonen untersucht. Für jeden Messpunkt wird die Thermospannung ermittelt. Die Flammenform wird zusätzlich skizziert und mit den Messwerten ergänzt. Die Umrechnung in Temperatur erfolgt anhand der für den verwendeten Thermoelementtyp gültigen Eichtabelle.
