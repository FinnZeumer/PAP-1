\chapter{Durchführung}

\section{Versuchsaufbau}

\subsection*{Aufgabe 1: Eichung der Thermometer bei 0\,°C}
Für die Eichung wird ein Pt100-Thermometer mit Adapterbox verwendet. Die vier Anschlussleitungen können über 4\,mm-Buchsen abgegriffen werden. Eine Stromquelle liefert den Messstrom von 1\,mA. Ein Voltmeter dient zur Spannungsmessung und kann wahlweise an die Buchsen der Stromquelle (Zweileiterschaltung) oder direkt an die Adapterbox (Vierleiterschaltung) angeschlossen werden.  
Zur Erzeugung der Referenztemperatur von 0\,°C wird ein Becherglas mit zerkleinertem Eis und Wasser befüllt. Der Glasballon mit dem Messsensor wird vollständig in das Gemisch eingetaucht. Zusätzlich stehen ein Pyrometer zur Messung der Oberflächentemperatur sowie ein Flüssigkeitsthermometer zur Verfügung.

\subsection*{Aufgabe 2: Temperaturmessung bis 100\,°C}
Das Wasserbad wird mit einer Heizplatte erhitzt. Zur Temperaturhomogenisierung wird ein Rührmechanismus verwendet. Als Messgeräte dienen das Pt100-Thermometer, das Gasthermometer und ein Pyrometer. Der Umgebungsdruck wird mit einem Barometer bestimmt.

\subsection*{Aufgabe 3: Temperaturmessung bei tiefen Temperaturen}
Ein Dewargefäß wird wahlweise mit einer Trockeneis-Alkohol-Mischung oder mit flüssigem Stickstoff befüllt. Der Glasballon mit Messsensor wird in das jeweilige Kühlmedium eingebracht. Für die Messung stehen das Pt100-Thermometer sowie das Gasthermometer zur Verfügung.

\subsection*{Aufgabe 4: Temperaturmessung mit dem PtRh-Thermoelement}
Zur Untersuchung hoher Temperaturen wird ein Gasbrenner mit regelbarer Luftzufuhr genutzt. Das PtRh-Thermoelement (Typ S oder Typ B) wird in verschiedene Bereiche der Flamme eingeführt. Die Temperaturbestimmung erfolgt über die gemessene Thermospannung unter Zuhilfenahme der passenden Eichtabelle.

\section{Messverfahren}


Nach dem Aufbau der Zwei- und Vierleiterschaltung werden beide Varianten ausprobiert. Bei stabilisierter Temperatur im Wasser-Eis-Gemisch wird die Spannung des Pt100, der Druck des Gasthermometers sowie die Pyrometertemperatur aufgezeichnet. Das Flüssigkeitsthermometer wird als zusätzliche Kontrolle herangezogen. Maßgeblich ist das Minimum der Pt100-Spannung als Eichpunkt.


Das Wasser wird stufenweise erhitzt. Beginnend bei etwa 10\,°C werden in Schritten von ca. 10\,°C die Pt100-Spannung, der Gasthermometerdruck und die Pyrometertemperatur aufgenommen. Als letzter Messpunkt dient die Temperatur des siedenden Wassers. Der Luftdruck wird parallel am Barometer erfasst. Das Pyrometer wird schräg auf die Wasseroberfläche gerichtet, um Verfälschungen durch Wasserdampf zu vermeiden.


Zunächst wird die Trockeneis-Alkohol-Mischung genutzt. Nach ausreichender Abkühlung und Temperaturstabilisierung werden Pt100-Spannung und Druck protokolliert. Anschließend wird der Versuch mit flüssigem Stickstoff wiederholt. Der Glasballon wird vollständig eingetaucht und die Messwerte nach Abklingen der starken Verdampfung notiert. Das Pyrometer wird nicht eingesetzt, da es bei diesen Temperaturen unbrauchbar ist.

\subsection*{Aufgabe 4: Temperaturmessung mit dem PtRh-Thermoelement}
Das Thermoelement wird in die Flamme des Gasbrenners eingeführt. Bei starker und schwacher Luftzufuhr werden jeweils mehrere Messpunkte in unterschiedlichen Flammenzonen untersucht. Für jeden Punkt wird die Thermospannung bestimmt. Die Flammenform wird zusätzlich skizziert und mit den Messwerten ergänzt. Die Umrechnung in Temperatur erfolgt anhand der für den verwendeten Thermoelementtyp gültigen Eichtabelle.
