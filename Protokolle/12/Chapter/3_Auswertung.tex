\onecolumn
\chapter{Auswertung}
% Liste der genutzer Formeln für die Fehlerrechnung
\section*{Fehlerrechnung}
Für die statistische Auswertung von $n$ Messwerten $x_i$ werden folgende Größen definiert \cite{errorSkript25}:
\begin{align}
    \bar{x} &= \frac{1}{n} \sum_{i=1}^{n} x_i \vphantom{\sqrt{\sum_i^n}^2} && \text{\textcolor{gray}{Arithmetisches Mittel}} \label{eq:arithmetisches_mittel} \\
    \sigma^2 &= \frac{1}{n-1} \sum_{i=1}^{n} (x_i - \bar{x})^2 \vphantom{\sqrt{\sum_i^n}^2} && \text{\textcolor{gray}{Variation}} \label{eq:variation} \\
    \sigma &= \sqrt{\frac{1}{n-1} \sum_{i=1}^{n} (x_i - \bar{x})^2} \vphantom{\sqrt{\sum_i^n}^2} && \text{\textcolor{gray}{Standardabweichung}} \label{eq:standardabweichung} \\
    \Delta \bar{x} &= \frac{\sigma}{\sqrt{n}} = \sqrt{\frac{1}{n(n-1)} \sum_{i=1}^n(\bar x - x_i)^2} \vphantom{\sqrt{\sum_i^n}^2} && \text{\textcolor{gray}{Fehler des Mittelwerts}} \label{eq:fehler_mittelwert} \\
    \Delta f &= \sqrt{\left(\frac{\partial f}{\partial x} \Delta x\right)^2 + \left(\frac{\partial f}{\partial y} \Delta y\right)^2} \vphantom{\sqrt{\sum_i^n}^2} && \text{\textcolor{gray}{Gauß’sches Fehlerfortpflanzungsgesetz für $f(x,y)$}} \label{eq:gauss_fehlfortpflanzung} \\
    \Delta f &= \sqrt{(\Delta x)^2 + (\Delta y)^2} \vphantom{\sqrt{\sum_i^n}^2} && \text{\textcolor{gray}{Fehler für $f = x + y$}} \label{eq:fehler_summe} \\
    \Delta f &= |a| \Delta x \vphantom{\sqrt{\sum_i^n}^2} && \text{\textcolor{gray}{Fehler für $f = ax$}} \label{eq:fehler_proportional} \\
    \frac{\Delta f}{|f|} &= \sqrt{\left(\frac{\Delta x}{x}\right)^2 + \left(\frac{\Delta y}{y}\right)^2} \vphantom{\sqrt{\sum_i^n}^2} && \text{\textcolor{gray}{relativer Fehler für $f = xy$ oder $f = x/y$}} \label{eq:relativer_fehler} \\
    \sigma &= \frac{|a_{lit} - a_{gem}|}{\sqrt{\Delta a_{lit}^2 + \Delta a_{gem}^2}} \vphantom{\sqrt{\sum_i^n}^2} && \text{\textcolor{gray}{Berechnung der signifikanten Abweichung}} \label{eq:signifikante_abweichung}
\end{align}

\twocolumn

\section{Bestimmung des Richtmomentes}
Kommen wir also nun zur Auswertung der Aufgaben. Dafür Beginnen wir damit, die Werte der Tabelle 1  aus dem \hyperref[Protokoll]{Protokoll}.
Dabei ist $x$ die Winkelauslenkung der Aluminiumscheibe, $m$ hängende Masse, $F$ die Gewichtskraft mit $\left| g \right| =9,81 \frac{m}{s^2}$, die auf die Masse wirkt $M$ das berechnete Drehmoment nach \hyperref[eq:gleichgewichts_zustand]{Gleichung 2.1} und $\Delta M$ seine Ungenauigkeit nach der \hyperref[eq:gauss_fehlfortpflanzung]{Gauß'schen Fehlerfortpflanzung}:
\begin{equation}
    \Delta M = \left| m \right| \cdot \left| g \right| \cdot \Delta r
\end{equation}
Dabei definieren die 0g den Startpunkt. 0g ist physikalisch in dem Kontest natürlich unsinnig und meint eigentlich die Startmasse, die hier nur aus dem Massenteller besteht, aber keine zusätzliche Masse.
Der Radius der Aliminumplatte entspricht dabei $r = 10,000 \pm 0,005 cm$, also ist $\Delta r = 0,005$, was der Ungenauigkeit der Schieblehre entspricht.
\begin{table}[h!]
    \begin{tabular}{c | c | c | c }
    $m [g]$ & $\varphi [^{\circ}]$ & $M \, [10^{-3}Nm]$ & $\Delta M \, [10^{-6}Nm]$ \\
    \hline
    0   & 0   & ---     &  ---    \\
    50  & 60  &  4,9050 &  2,4525 \\
    100 & 122 &  9,8100 &  4,9050 \\
    150 & 180 & 14,4150 &  7,3575 \\
    200 & 242 & 19,6200 &  9,8100 \\
    250 & 302 & 24,5250 & 12,2625 \\
    300 & 366 & 29,4300 & 14,7150 \\
    \hline
    \end{tabular}
    \caption{Messungen der Rotationsauslenkung der Aluminumscheibe und die berechneten Drehmomente.}
    \label{tab:verschiedene_massen_winkel_auslenkung}
\end{table}

Stellen wir nun \hyperref[eq:kraft_drehmoment_zusammenhang]{Gleichung 1.3} um, so kommen wir auf:
\begin{equation}
    D = - \frac{M}{\varphi}
\end{equation}

Daher plotten wir als nächstes das Drehmoment $M$ gegen den Auslenkungswinkel $\varphi$ und berechnen seine Steigung $m$, welche dem Drehmoment entspricht, nach
\begin{equation}
    m = \frac{\Delta M}{\Delta \varphi}
\end{equation}

Dies kann der \hyperref[]{Abbildung zur Bestimmung des Richtmomentes} entnommen werden: 

\begin{align}
    m_A = \frac{}{} \\
    m_F = \frac{(24,5290) [Nm]}{(310) [^\circ]} = 0,0791
\end{align}

Dabei ist $m_A$ die Steigung der Ausgleichsgeraden und $m_F$ die Steigung der Fehlergeraden, welche über das Min.-Max.-Verfahren bestimmt wurde. $\Delta \varphi$ sind dabei $2^\circ$.

\section{Bestimmung des Richtmomentrs über Trägheitsmoment}

\begin{table}[h!]
    \begin{tabular}{c | c | c | c | c}
    Scheibe & $t \, [s]$& $T \, [s]$ & $\bar{T} \, [s]$ & $\Delta \bar{T} [s]$\\
    \hline
    Keine       & 23,09 & 1,155 & 1,162 & 0,01\\
                & 23,31 & 1,166 & $= T_1$ & \\
                & 23,28 & 1,164 &  & \\
     \hline
    Messing-    & 34,89 & 1,745 & 1,741 & 0,01\\
    Platte      & 34,75 & 1,738 & $= T_2$ & \\
                & 34,78 & 1,739 &  & \\
    \hline
    \end{tabular}
    \caption{Messungen der Schwingdauer einer regelmäßigen Messingplatte unter 20 Schwingungen.}
    \label{tab:regelmäßige_messingplatte}
\end{table}

Den Fehler der der Periodendauer $\Delta \bar{T}$ wurde über eine Reaktionszeit von $0,2s$ über \hyperref[eq:gauss_fehlfortpflanzung]{Gauß'sche Fehlfortpflanzung} berechnet sich der Fehler zu:
\begin{equation}
    \Delta \bar{T} = 0,20s \cdot \frac{1}{20} = 0,01s
\end{equation}

Für die weitere Berechnung brauchen wir die Werte der Messingscheibe:
\begin{align*}
    \text{Durchmesser: }         d_M   &= 110mm    &&\pm 0,005mm \\
    \Rightarrow \text{Radius: }  r_M   &= 55mm     &&\pm 0,0025mm \\
    \text{Masse: }               m_M   &= 646g     &&\pm 1g
\end{align*}

\begin{table}[h!]
    \begin{tabular}{c | c | c | c}
    Achse & d $[mm]$ & t $[s]$ & I $[kg \cdot m^2]$ \\
    \hline
    $a_0$ & 0,0 & 44,42 & \\ 
    $a_1$ & 0,5 & 44,58 & \\
    $a_2$ & 1,0 & 44,73 & \\
    $a_3$ & 1,5 & 45,10 & \\
    $a_4$ & 2,0 & 45,90 & \\
    $a_5$ & 2,5 & 47,67 & \\
    \hline
    \end{tabular}
    \caption{Messung des Trägheitsmomente verschiedener Drehachsen einer unregelmäßigen Messingplatte.}
    \label{tab:unregelmäßige_messingplatte}
\end{table}