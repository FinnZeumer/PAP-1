\chapter{Durchführung}

\section{Versuchsaufbau}

\subsection*{Genauigkeit der Messgeräte}
\begin{table}[h!]
    \centering
    \begin{tabular}{l | l | c}
    Gerät & Präzision & Ungenauigkeit \\
    \hline
    Stoppuhr        & 0,01s     & / / / \\
    Waage           & 1g        & 1g \\
    Schieblehre     & 1mm       & 0,05mm \\
    Al-Teller       & 2 Grad    & / / / \\
    \hline
    \end{tabular}
    \caption{Genauigkeit der benutzen Geräte \cite{Stoppuhr,Waage}}
    \label{tab:genauigkeit_der_geräte}
\end{table}

Der Versuch bestand aus 5 Unteraufgaben. Alle diesen der Bestimmung rotatorischer Eigenschaften. 
Darunter die Bestimmung des Richtmoments und später des Stein'schen Satzes.
\subsection*{Aufgabe 1) Bestimmung des Richtmomentes}
Wir benutzen den Drehtisch und legen die Aluminiumscheibe mit der Grad-Skala drauf.
Diese hat eine Befestigung für die Schnurnut. Diese hängt über eine Rolle vom Tisch herunter. 
An dieser Schnurnut hängt der Massenteller, seine Auslemkung wurde auf 0 Grad gestellt.
Dannach wurden die 6 50g Massen an den Teller gehängt und die jeweilige Auslenkung bzw. Rotation dokumentiert.
Da die scheibe jedcoh mehr als 360 Grad gedreht wird, muss nach Messung vier der Drehtisch selbst wieder gedreht werden, 
damit die Auslenkung normal möglich ist. Hierfür wurde dann die 200g + Massenteller als 0 Grad gesetzt.

\subsection*{Aufgabe 2) Bestimmung des Richtmpments via bekanntem Trägheitsmoment}
In der zweiten Aufgabe wurde der Alluminumteller mit der regelmäßigen/symmetrischen Messingplatte ausgetauscht. 
Ihr Drehmoment lässt sich leicht berechnen, da die Formel zur Berechnung bekannt ist; 
benötigt werden jedoch sein Radius und seine Masse, diese werden gemessen. 
Anschließend wird der Drehtisch dreimal ohne Messingplatte und drei Mal mit Messingplatte gleichweit ausgelengt 
und seine Schwindauer für 20 Schwinungen per Hand gestoppt. 

\subsection*{Aufgabe 3) Schwerpunkt-Bestimmung}
In dieser Aufgabe musste ledeglich der Schwerpunkt einer unregelmäßigen Messingscheibe bestimmt werden.
Dafür haben wir eine Schneide, auf der die Messingplatte balancierd wird, da wo die Platte (annährend) im Gleichgewicht ist,
wird die Schneide auf des Schwerpunktes sein. Hier wird eine Line gezogen. Dies wiederholt man aus einem anderen Winkel ein zweites Mal.
Es wird sich ein gezeichnetes Kreuz bilden, an desssen Mittelpunkt zugleich der Schwerpunkt der unregelmäßigen Messingplatte ist.

\subsection*{Aufgabe 4 + 5) Steinsch'er Satz}
Die letzen zwei Aufgaben dienen dazu, den Stein'schen Satz zu zeigen. Dafür werden auf den unregelmäßigen Messingkörßer
5 weitere Makierungen gesetzt, die auf einer der Geraden auf der Messingplatte liegen. Sie werden all im Abstand von 0,5cm gesetzt, startend vom Schwerpunkt.
Es sind nun insgesamt 6 Makierungen auf dem unregelmäßigen Messingkörper. 

Nun wird die Messingpaltte auf den Drehtisch fixiert. 
Das Ziel ist es, für alle Makierungen wieder die Schwingdauer für 20 Schwingungen zu bestimmen. 
Die Werte für alle Schwingdauern werden dokumentiert und dann die jeweiligen Trägheitsmomente bestimmt.

\section{Messverfahren}
