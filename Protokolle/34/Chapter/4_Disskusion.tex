\chapter{Diskussion}

\section{Zusammenfassung}  
Im Rahmen des Versuhces wurde die molare Extinktion des Kaliumpermanganats sowohl nach dem Lambert'schen als ach nach dem Beer'schen Verfahren bestimmt.  
Aus der Lambert Gerade ergab sich für den dekadischen Absorptionskoeffizienten ein Wert von  
\[
k' = (0{,}0971 \pm 0{,}0106)\,\mathrm{cm^{-1}}.
\]  
Daraus wurde unter Verwendung der bekannten Konzentration \( c = 5 \cdot 10^{-5}\,\mathrm{mol/l} \) der molare Extinktionskoeffizient zu  
\[
\epsilon_L = (1900 \pm 200)\,\mathrm{cm^2/mol}
\]  
bestimmt.  

Das Beer'sche verfahren ergab hingegen eine Steigung von  
\[
m = -(7000 \pm 5000)\,\mathrm{l/mol},
\]  
woraus sich bei einer Schichtdicke von \( l = 1{,}5\,\mathrm{cm} \) der molare Extinktionskoeffizient  
\[
\epsilon_B = (5000 \pm 3000) \cdot 10^3\,\mathrm{cm^2/mol}
\]  
ergab.  

Der Vergleich der beiden Ergebnisse führte zu einer Signifikanten Abweichung von  
\[
1{,}03\sigma,
\]  
wobei der nach dem Beer'schen Verfahren ermittelte Wert etwa dopelt so groß ist wie der nach dem Lambert'schen Verfahren bestimmte.  

\section{Diskussion}  
Die beiden Methoden zur Bestimmung der molaren Extinktion führen zu Deutlich unterschiedlichen Resultaten. der Wert aus der Beer'schen Methode liegt ungefähr um den Faktor zwei über dem Wert aus der Lambert'schen Auswertung. Da die Abweichung trotz dieses Unterschieds nur etwa \( 1\sigma \) beträgt, ist keine signifikante Diskrepanz im statistischen Sinn festzustellen. Dies liegt in erster Linie an den sehr großen Unsicherheiten, insbesondere beim Beer'schen Verfahren, welche die Streuung der Messpunkte stark erhöhen und den Fehler der Ausgleichsgeraden vergrößern.  

Der höhere Wert von \( \epsilon_B \) könnte auf systematische Fehler in der Konzentrationsbestimmung oder auf ungenaue Volumenmessungen bei der sukzessiven Verdünnung zurückzuführen sein. Ebenso ist nicht auszuschließen, dass beim Mischen der Lösungen keine vollständige Homogenisierung erreicht wurde, wodurch die effektive Konzentration von der theoretischen Berechnung abweichen kann.  

Insgesamt deuten die Ergebnisse darauf hin, dass die Abweichung primär auf experimentelle Ungenauigkeiten und weniger auf eine Verletzung der theoretischen Beziehungen von Lambert und Beer zurückzuführen ist.  

\section{Kritik}  
Die größte Unsicherheitsquelle des gesamten Versuches lag eindeutig in der graphischen Auswertung. Das Ablesen der Steigungen auf halblogarithmischem Papier ist mit einer erheblichen subjektiven Komponente verbunden und führt zu großen Fehlerintervallen. Eine digitale Auswertung der Messpunkte mittels Regressionsanalyse hätte die Genauigkeit der Bestimmung deutlich verbessert.  

Weiterhin war die Intensitätsmessung stark von der korrekten Einstellung der Integrationszeit abhängig. Eine unzureichende Anpassung führte leicht zu Sättigungseffekten, insbesondere bei den geringeren Schichtdicken und Konzentrationen. Auch die Korrektur der Intensität nach Gleichung (\ref{eq:correction}) brachte zusätzliche Unsicherheiten ein, da kleine Messfehler bei den Blendenmaßen quadratisch in die Berechnung eingehen.  

Trotz dieser Limitierungen stimmen die Ergebnisse qualitativ mit den theoretischen Erwartungen überein. Der Verlauf der Absorption zeigt das charakteristische Verhalten des Kaliumpermanganats, und beide Methoden liefern Extinktionskoeffizienten in derselben Größenordnung. Die verbleibende Diskrepanz ist angesichts der großen experimentellen Fehler vertretbar, weist jedoch darauf hin, dass bei zukünftigen Messungen eine präzisere graphische und rechnerische Auswertung zwingend erforderlich ist.
