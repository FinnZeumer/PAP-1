\onecolumn
\chapter{Auswertung}
\label{ch:auswertung}

\section*{Fehlerrechnung}
Für die statistische Auswertung von $n$ Messwerten $x_i$ werden folgende Größen definiert \cite{errorSkript25}:
\begin{align}
    \bar{x} &= \frac{1}{n} \sum_{i=1}^{n} x_i \vphantom{\sqrt{\sum_i^n}^2} && \text{\textcolor{gray}{Arithmetisches Mittel}} \label{eq:arithmetisches_mittel} \\
    \sigma^2 &= \frac{1}{n-1} \sum_{i=1}^{n} (x_i - \bar{x})^2 \vphantom{\sqrt{\sum_i^n}^2} && \text{\textcolor{gray}{Variation}} \label{eq:variation} \\
    \sigma &= \sqrt{\frac{1}{n-1} \sum_{i=1}^{n} (x_i - \bar{x})^2} \vphantom{\sqrt{\sum_i^n}^2} && \text{\textcolor{gray}{Standardabweichung}} \label{eq:standardabweichung} \\
    \Delta \bar{x} &= \frac{\sigma}{\sqrt{n}} = \sqrt{\frac{1}{n(n-1)} \sum_{i=1}^n(\bar x - x_i)^2} \vphantom{\sqrt{\sum_i^n}^2} && \text{\textcolor{gray}{Fehler des Mittelwerts}} \label{eq:fehler_mittelwert} \\
    \Delta f &= \sqrt{\left(\frac{\partial f}{\partial x} \Delta x\right)^2 + \left(\frac{\partial f}{\partial y} \Delta y\right)^2} \vphantom{\sqrt{\sum_i^n}^2} && \text{\textcolor{gray}{Gauß’sches Fehlerfortpflanzungsgesetz für $f(x,y)$}} \label{eq:gauss_fehlfortpflanzung} \\
    \Delta f &= \sqrt{(\Delta x)^2 + (\Delta y)^2} \vphantom{\sqrt{\sum_i^n}^2} && \text{\textcolor{gray}{Fehler für $f = x + y$}} \label{eq:fehler_summe} \\
    \Delta f &= |a| \Delta x \vphantom{\sqrt{\sum_i^n}^2} && \text{\textcolor{gray}{Fehler für $f = ax$}} \label{eq:fehler_proportional} \\
    \frac{\Delta f}{|f|} &= \sqrt{\left(\frac{\Delta x}{x}\right)^2 + \left(\frac{\Delta y}{y}\right)^2} \vphantom{\sqrt{\sum_i^n}^2} && \text{\textcolor{gray}{relativer Fehler für $f = xy$ oder $f = x/y$}} \label{eq:relativer_fehler} \\
    \sigma &= \frac{|a_{lit} - a_{gem}|}{\sqrt{\Delta a_{lit}^2 + \Delta a_{gem}^2}} \vphantom{\sqrt{\sum_i^n}^2} && \text{\textcolor{gray}{Berechnung der signifikanten Abweichung}} \label{eq:signifikante_abweichung}
\end{align}

\twocolumn


% /////////////// Aufgabe 1 ///////////////
\section{Aufgabe 1: Qualitative Beobachtung der drei Zylinder}
Die Messung zeigte, dass beim gleichzeitigen Start der drei Rollkörper der Verbundzylinder 
zuerst und der Hohlzylinder zuletzt das Ende der schiefen Ebene erreichte. 
Ursache hierfür ist das unterschiedliche Trägheitsmoment der Körper. 
Da alle nahezu die gleiche Masse besitzen, entscheidet die Massenverteilung über das Trägheitsmoment: 
Beim Hohlzylinder liegt die Masse weiter von der Rotationsachse entfernt, wodurch sein 
Trägheitsmoment gemäß \hyperref[eq:traegheit_hohl]{Gleichung \ref*{eq:traegheit_hohl}} größer ist. 
Beim Verbundzylinder ist die Masse hingegen näher an der Achse konzentriert, was zu einem kleineren Trägheitsmoment als beim Vollzylinder führt 1
(vgl. \hyperref[eq:traegheit_voll]{Gleichung \ref*{eq:traegheit_voll}}).

% /////////////// Aufgabe 2 ///////////////

\section{Aufgabe 2: Bestimmung der Rollbeschleunigung}
Zunächst werden die wichtigen Maße der Ebene vermerkt. Die Zylinder beschleunigen dabei über eien Strecke von $l_{r} = (87,20 \pm 0,05) \, cm$. Die untere Kante der Abrollhöhe liegt bei $h` = (12,70 \pm 0,05) cm$. Wi müssen jedoch noch die Decke der Holzplatte und der Metallplatte dazu addieren. Die Werte liegen bei $D_h= (1,8 \pm 0,05) cm$ für die Holzplatte und $D_m= (0,2 \pm 0,05) cm$ für die Metallplatte.
Der Fehler der gesamt Starthöhe ist nach \hyperref[eq:gauss_fehlfortpflanzung]{Gauß'scher Fehlerfortpflanzung (\ref*{eq:gauss_fehlfortpflanzung})}
\begin{equation}
    \Delta h_s = 0,09 cm
\end{equation}

Unsere Starthöhe liegt somit bei
\begin{equation}
    h_s = (14,70 \pm 0,09)\, cm.
\end{equation}

Um später Ergebnisse besser überprüfen zu können, haben wir auch noch die Projektion der Lauffläche (die Ankathete) vermessen und kamen dabei auf eine Länge von $l_{an} = (86,00 \pm 0,05) \, cm$.
