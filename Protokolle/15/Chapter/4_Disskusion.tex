\chapter{Disskusion}
\label{ch:diskussion}

\section{Zusammenfassung}
\subsection{Aufgabe 1}
Der Vollzylinder ist schneller als der Hohlzylinder. Der Verbundszylinder wiederrum ist schneller als der Vollzylinder. Je näher die Masse am Zentrum, desto schneller beschleunigt der Körper (s. Aufgabe 4).

\subsection{Aufgabe 2}
In dieser Aufgabe wurde über die vermessene Laufstrecke und Starthöhe, sowie die Projektion der Laufstrecke auf den Tisch, der Winel $\alpha$ bestimmt. Der Wert für die weiteren Berechnugen war dabei:
\begin{equation}
\boxed{
    \alpha_{\cos} = (9,5163 \pm 0,0229) ^\circ
}.
\end{equation}
Dieser deckt sich hervorragend mit $\alpha_{\sin}$, jedoch wurde der Kosinus bestimmte Wert verwendet, da sein Fehler kleiner ist. Einen Großen Unterschied hätte es jedoch nicht gemacht, den anderen Winkel zur Berechnung zu benutzen.
Aus diesem Winkel wurden die vermuteten Beschleunigungen des Hohl- und Vollzylinders bestimmt:
\begin{equation}
    \boxed{
        a_{voll,lit} = (1,081 \pm 0,003) \frac{m}{s^2}
    }.
\end{equation}

\begin{equation}
    \boxed{
        a_{hohl,lit} = (0,92 \pm 0,05) \frac{m}{s^2}
    }.
\end{equation}

Die Werte decken sich auch mit der Beobachtung, denn der Vollzylinder beschleunigt schneller, sodass dieser auch schneller Unten ist (s. AUfgabe 1).

\subsection{Aufgabe 3}
In dieser Aufgabe wurden mit Hilfe der Messwerte die Beschleunigung der Zylinder bestimmt. Die Ergebnisse sind folgende:
\begin{equation}
    \boxed{
        a_{voll} = (0,9764 \pm 0,027) \frac{m}{s^2}
    }
\end{equation}
und
\begin{equation}
\boxed{
    a_{hohl} = (0,7012 \pm 0,0125) \frac{m}{s^2}
}.
\end{equation}

Diese Werte Decken sich wieder mit der Beobachtung aus Aufgabe 1, aber sind unter den Werten aus Aufgabe 2.

\subsection{Aufgabe 4}
In dieser Aufgabe wurde die Energieerhaltung anhand von rollenden Zylindern bestimmt. Dabei wurden in dieser Rechnung lediglich 
\begin{equation}
    \frac{E_{rot,voll} + E_{trans,voll} }{E_{pot,voll}} = 25,5\%.
\end{equation}
der potentiellen Energie des Vollzylinders und 
\begin{equation}
    \frac{E_{rot,hohl} + E_{trans,hohl} }{E_{pot,hohl}} = 24,2\%.
\end{equation}
der potentiellen Energie des Hohlzylinders erhalten. 


\section{Disskusion}
\subsection{Aufgabe 2 und 3}
Im Aufgabenteil 3 trat das Problem auf, dass die Beschleunigungen, die mit den Messdaten berechnet wurden, nichst rechnerisch stringent mit den Werten der theoretischen Berechnug aus Aufgabe 2 sind. Die in Aufgabe zwei berechneten Werte lassen sich als soll-Wert interpretieren, welcher sich in Idealbedinugen einstellen sollte. Die graphisch bestimmten Werte weichen jedoch enorm ab. Die Huaptfehlerquelle wird sher wahrscheinlich die Methodig der Bestimmung selbst sein, graphische Bestimmungen sind immer sehr fehleranfällig und es treten zeichen Ungenauigkeiten neben den Messungenauigkeiten zusätzlich auf. Dies ist die Huaptfehlerquelle. Zusätzlich treten immer ungewollte Energie Verluste über nicht berücksichtigte Reibungen und Wärmeverlust auf. Somit werden die tatsächlichen Werte immer unter den soll-Werten liegen. Die Bestimmung des Winkels wird wohl kein großer Fehler hier sein, da dieser auf zwei Methoden wegen bestimmt wurde, die sich beide statistsich decken. 

\subsection{Aufgabe 4}
In diesem Aufgabenteil trat ein Problem auf, die Umgesetzte Energie der potentiellen Energie ist weit unter der Erwartung. Realistisch wäre ein Wert von definiert über 80\% umgesetzter Energie gewesen. Es fällt jedocvh auf, dass das Problem chronisch ist, denn für den Hohl- und den Vollzylinder ist die Umgeseätze Energie in derselben Größenordnung (~25\%). 
Darüber hinaus lässt sich der Fehler eindämmen, denn es wird entweder dei potentielle Energie überschätzt, oder die kinetische unterschätzt. Es sollen einmal beide Seiten verargumentiert werden.
\subsubsection*{Überschätzte potentielle Energie}
ür diese These spricht, dass die Höhe falschg bestimmt wurde. Dies ist auch der einzige Wert, der falsch bestimmt sein könnte, die restlichen Bestandteile sind Literaturwerte. Gehen wir davon aus, dass die kinetische Energie korrekt bestimmt wurde, dann kann amn aus $E_{kin} = E_{pot}$ auf h umstellen:
\begin{equation}
    h = \frac{E_{kin}}{mg}
\end{equation}

Somit bekommt man die Höhen
\begin{align}
h_{voll} &= 3,24 cm \\
h_{hohl} &= 3,07 cm.
\end{align}

Die Höhe müsste also bei $\bar h = 3,16 cm$ leigen. Ein so starker Messfehler ist unglaublich unwarhscheinlich, zudem können wir die Höhe über die anderen vermessenen Kanten bestätigen.

Als Resultat wird der Fehler wahrscheinlich bei der kinetischen Energie liegen.

\subsubsection*{Unterschätzung kinetische Energie}
Mögliche Fehlerquellen der kinetischen Energie sind die Geschwindigkeit und für den Hohlzylinder die Radien. Das die Radien der Fehler sind, ist jedcoh sehr unwarhscheinlich, da der Fehler chronisch ist und der Radius für den Vollzylinder nicht relevant ist. Daher wird der Fehler vermutlich in der Geschwindigkeit leigen. 
Diese wiederrum wurde aus einer Strecken- und einer Zeitmessung bestimtm. Die Zeit wird auch nur unwahrscheinlich der Fehler sein, zwar wurde diese für beie Methoden gleich bestimmt und könnte somit chronische Fehler verursachen, jedoch sind die Einzelmessungen alle konsitent. Es müsste also ein starker maschineller Fehler sein, damit dies der Fehler sein kann. Zudem kommt, dass die Strecke die wahrscheinlichste Fehlerquelle ist, wie nun erlätert werden soll.
Die Rolldistanz auf der flachen Ebene wurde mit $s = 16cm$ bestimmt.  Wir gehen in der folgenden Rechnung von einem >>wahren<< Energiesatz der potentiellen Energie aus. Dann kann man die Energien gleichsetzen und nach s auflösen:
Für den Vollzylinder
\begin{equation}
    s_{voll} = \sqrt{\frac{4 \cdot E_{pot} \cdot t^2}{3 \cdot m}} = 31,67 cm
\end{equation}
und für den Hohlzylinder
\begin{equation}
    s_{hohl} = \sqrt{\frac{E_{pot} \cdot t^2}{\frac{m}{2}+\frac{m}{4}(\frac{r^2}{R^2}+1)}} = 32,53 cm
\end{equation}

Im Mittel müsste die Strecke gerade $32,1 cm$ sein, also das Doppelete der eingestellten Länge. Das dies die tatsächliche Strecke ist, ist dadurch plausibel, dass immer in 16cm Abständen gemessen wurde, offensichtlich wurde hier misskommuniziert, wodurch die tatsächliche Strecke wahrscheinlich bei ~32cm lag. Visuell lässt sich diese These weiter durch \hyperref[fig:versuchsaufbau_4]{Abbildung \ref*{fig:versuchsaufbau_4}} sehen, da dies gerade der benutze Versuchsaufbau ist. Über grobe Abschätzungen ist schnell festzustellen, dass die Distanz der Lichtschranken LS1 und LS2 bei 64cm liegt. Dass die Lichtschranken LS3 und LS4 eher halbsoweit außeinanderstehen und nicht ein viertel so weit lässt sich gut abschätzen. 

\subsubsection*{Korrigierte Werte}
Wir wollen nun alle Energien nochmal mit der Strecke von $s = 32cm$ bestimmen. Wir werden werden den Fehler hier glecihgroß lassen, da wir (wenn die These stimmt) mit derselben Präzision gemessen haben. Somit kommt man auf Energien für den Vollzylinder von
\begin{align}
    v_{voll} = (1,3019 \pm 0,0028) m/s \\
    E_{pot,voll} = (0,5532 \pm 0,0025) J \\
    E_{trans,voll} = (0,3763 \pm 0,0018) J \\
    E_{rot,voll} = (0,1881 \pm 0,0009) J \\
    \text{Abweichung} = 102\%
\end{align}


\begin{align}
    v_{hohl} = (1,151079 \pm 0,002225) m/s \\
    E_{pot,hohl} = (0,552 \pm 0,004) J \\
    E_{trans,hohl} = (0,2935 \pm 0,0013) J \\
    E_{rot,hohl} = (0,2407 \pm 0,0011) J \\
    \text{Abweichung} = 97,9\%
\end{align}

Das die Abweichungen nicht vollkommen sinnvoll sind, liegt daran, dass die Stecke im Nachhinein berechnet wurde. Die Ergebnisse sind nicht vollkommen sinnvoll, vermutlich ist die tatsächliche Strecke eigentlich etwas kleiner als die 32cm, aber ca. diese Größe.

\subsubsection*{Fazit der Aufgabe}
Zusätzlich zeitgt die Aufgabe wunderbar, wieso der Vollzylinder schneller ist, als der Hohlzylinder. Denn beide starten mit (circa) dersleben potentiellen Startenergie. Jedoch wandelt der Hohlzylinder mehr seiner Energie in Rotationsenergie um, dies resultiert daher, dass mehr Masse am Rand des Zylinders auch bedeutet, dass dort eben mehr Masse ins Rollen gebracht werden muss, daher muss der Körper einen größeren Teil seiner Energie ins Beschleunigung dieser Masse stecken.


\section{Kritik}
Wie bei allen Experimenten des physikalischen Anfängerpraktikums, werden hier viele Überidealisierungen getroffen, wodurch sich chronisch ein Fehler durch alle Rechnungen zeiht. Die meisten davon sind jedoch gegenüber der Messgenauigkeiten vernachlässigbar. Um genauere Ergebnisse zu bekommen, muss zunächst die Messgenauigkeit erhöht werden. Als nächster logischer Schritt wäre es mehr Messungen vorzunehmen, da so der statistische Fehler minimiert werden kann. Erst ab einer hohen Messpräzision und einer hohen quantität sind genauere physikalische Modelle hier angebracht.