\chapter{Einleitung}

\section{Motivation}
Der Versuch 22 basiert auf dem von Robert A. Millikan entwickelten Prinzip zur Bestimmung der Elementarladung $e$. Millikan veröffentlichte 1913 in der Fachzeitschrift \textit{Physical Review} eine Arbeit mit dem Titel \textit{On the Elementary Electrical Charge and the Avogadro Constant} \cite{millikan1913}, für die er 1923 den Nobelpreis für Physik erhielt. 

Das Grundprinzip besteht in der Betrachtung der Kräfte, die auf ein elektrisch geladenes Teilchen (hier: Öltröpfchen) im homogenen Feld eines Plattenkondensators wirken. Die Elementarladung ist nach aktuellem Wissensstand die kleinste mögliche elektrische Ladung und von fundamentaler Bedeutung. Der Versuch ermöglicht es, aus der Fallgeschwindigkeit $v_f$ im feldfreien Raum und der Steiggeschwindigkeit $v_s$ im elektrischen Feld die Ladung $q$ eines einzelnen Öltröpfchens zu bestimmen. Die Quantelung elektrischer Ladung wird so experimentell nachgewiesen.

\section{Physikalische Grundlagen}
\cite{skript25}
Wir betrachten ein Öltröpfchen mit Radius $r$, Dichte $\rho_\text{Öl}$ und Geschwindigkeit $v$, das sich in Luft mit der Dichte $\rho_\text{Luft}$ bewegt. Es wirken folgende Kräfte:

\begin{align}
    F_G &= \frac{4}{3}\pi r^3 \rho_\text{Öl} g \label{eq:gewicht} \\
    F_A &= \frac{4}{3}\pi r^3 \rho_\text{Luft} g \label{eq:auftrieb} \\
    F_R &= 6 \pi r \eta v \label{eq:stokes} \\
    F_E &= q \frac{U}{d} \label{eq:elektrisch}
\end{align}

Hierbei bezeichnet $g$ die Schwerebeschleunigung, $\eta$ die Viskosität der Luft, $U$ die Spannung am Kondensator und $d$ den Plattenabstand. $q$ ist die Ladung des Öltröpfchens.

\subsection*{Feldfreier Raum}
Liegt keine Spannung am Kondensator an, wirkt keine elektrische Kraft. Das Kräftegleichgewicht für die konstante Fallgeschwindigkeit $v_f$ lautet nach \hyperref[eq:gewicht]{Gleichung \ref*{eq:gewicht}}, \hyperref[eq:auftrieb]{Gleichung \ref*{eq:auftrieb}} und \hyperref[eq:stokes]{Gleichung \ref*{eq:stokes}}:

\begin{equation}
    F_G - (F_A + F_R) = 0 \label{eq:fres_f}
\end{equation}

oder äquivalent:

\begin{equation}
    6 \pi r \eta v_f + \frac{4}{3}\pi r^3 \rho_\text{Luft} g = \frac{4}{3}\pi r^3 \rho_\text{Öl} g
\end{equation}

Daraus ergibt sich der Radius $r$ zu:

\begin{equation}
    r = \sqrt{\frac{9 \eta}{2 \rho g} v_f} \quad \text{mit } \rho = \rho_\text{Öl} - \rho_\text{Luft} \label{eq:radius}
\end{equation}

\subsection*{Mit elektrischem Feld}
Liegt Spannung an, wirkt zusätzlich die elektrische Kraft \hyperref[eq:elektrisch]{Gleichung \ref*{eq:elektrisch}}. Für die konstante Steiggeschwindigkeit $v_s$ ergibt sich:

\begin{equation}
    (F_E + F_A) - (F_G + F_R) = 0 \label{eq:fres_s}
\end{equation}

Aus der Differenzbildung von \hyperref[eq:fres_f]{Gleichung \ref*{eq:fres_f}} und \hyperref[eq:fres_s]{Gleichung \ref*{eq:fres_s}} folgt die Ladung $q$ des Tröpfchens:

\begin{equation}
    q = (v_f + v_s) \cdot \frac{6 \pi \eta d}{U} \cdot \sqrt{\frac{9 \eta}{2 \rho g} v_f} \label{eq:ladung}
\end{equation}

\subsection*{Korrektur der Viskosität}
Die Radien der Öltröpfchen liegen im Bereich $10^{-7}\,\text{m}$ bis $10^{-6}\,\text{m}$, was derselben Größenordnung wie die mittlere freie Weglänge von Luftmolekülen entspricht. Deshalb wird die Viskosität $\eta$ mit der sogenannten Cunningham-Korrektur versehen:

\begin{equation}
    \eta(r) = \eta_0 \left(1 + \frac{b}{rp}\right) \label{eq:cunningham}
\end{equation}

Hier ist $\eta_0$ der Grenzwert der Viskosität für große Tröpfchen, $p$ der Luftdruck und $b$ eine empirische Konstante. Da $r$ selbst von $\eta$ abhängt, müsste man \hyperref[eq:cunningham]{Gleichung \ref*{eq:cunningham}} in \hyperref[eq:radius]{Gleichung \ref*{eq:radius}} einsetzen, was zu einer quadratischen Gleichung führt. In der Praxis genügt es, mit $\eta_0$ zu rechnen. Der Fehler beträgt für $r$ etwa 5\,\%, für $f(r)$ nur 0,5\,\%.

\subsection*{Wichtige Konstanten}
Die im Versuch relevanten Konstanten sind in \hyperref[tab:konstanten]{Tabelle \ref*{tab:konstanten}} angegeben.

\begin{table}[h!]
\centering
\begin{tabular}{ll}
\hline
Bezeichnung & Wert \\
\hline
Viskosität der Luft & $\eta_0 = 1.81 \times 10^{-5}\,\text{Ns/m}^2$ \\
Schwerebeschleunigung & $g = 9.81\,\text{m/s}^2$ \\
Temperatur im Raum & $(25 \pm 0.5)^{\circ}\text{C}$ \\
Luftdruck im Raum & $(100460 \pm 10)\,\text{Pa}$ \\
Anliegende Spannung & $U_0 = 500\,\text{V}$ \\
Dichte des Öls bei 25$^{\circ}$C & $\rho_\text{Öl} = 877\,\text{kg/m}^3$ \\
Dichte der Luft & $\rho_\text{Luft} = 1.29\,\text{kg/m}^3$ \\
Konstante im Korrekturfaktor & $b = 7.78 \times 10^{-3}\,\text{Pam}$ \\
Abstand der Kondensatorplatten & $d = (6.00 \pm 0.05)\,\text{mm}$ \\
Skala auf dem Bildschirm & $1\,\text{Skt} = (5.00 \pm 0.13)\times 10^{-5}\,\text{m}$ \\
\hline
\end{tabular}
\caption{Wichtige Konstanten für den Versuch}
\label{tab:konstanten}
\end{table}
