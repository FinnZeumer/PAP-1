\chapter{Diskussion}

\section{Zusammenfassung}

Im Rahmen des Versuchs wurden die Spektrallinien der Quecksilber-, Helium- und Wasserstofflampen mittels eines Prismenspektrometers vermessen. Zunächst wurde die Eichkurve anhand der bekannten Wellenlängen des Quecksilberspektrums erstellt, um die Auslenkungen der Helium- und Wasserstofflampen später präzise bestimmen zu können. Die ermittelten Wellenlängen des Helium- und Wasserstoffspektrums zeigten nur geringe Abweichungen von den Literaturwerten, wobei die maximalen signifikanten Abweichungen in allen Fällen unter $2\sigma$ lagen, mit Ausnahme der roten Linie der Wasserstofflampe, die eine Abweichung von $5,2\sigma$ zeigte. Aus den Wellenlängen des Wasserstoffspektrums konnte die Rydberg-Konstante berechnet werden, die mit $\overline{R_\infty} = (1,11 \pm 0,17)\cdot 10^7 \, \mathrm{m^{-1}}$ praktisch mit dem Literaturwert übereinstimmt, wobei die Abweichung lediglich $0,06\sigma$ beträgt.

\section{Diskussion}

Die Messungen bestätigen die Diskretheit der Energieniveaus von Atomen, wie sie insbesondere im Wasserstoffatom durch die Balmer-Serie sichtbar wird. Die Eichkurve, die auf den Quecksilberlinien basiert, erwies sich als zuverlässig, um die Wellenlängen der Helium- und Wasserstofflampen zu bestimmen. Die geringen Abweichungen zwischen den gemessenen und den Literaturwerten lassen sich hauptsächlich durch den Ablesefehler des Nonius erklären, der die dominante Fehlerquelle in diesem Versuch darstellt. Die Genauigkeit der Messungen ist dabei insgesamt hoch, was sich in den kleinen $\sigma$-Abweichungen der meisten Linien widerspiegelt.  

Besonders die bläulichen und violetten Linien der Wasserstofflampe waren aufgrund ihrer geringen Intensität schwierig auszumessen, was sich in etwas größeren Unsicherheiten bei der Bestimmung der Wellenlängen äußert. Dennoch liegen die ermittelten Werte innerhalb der erwarteten Fehlerbereiche. Die Berechnung der Rydberg-Konstante ziegt, dass das Prismenspektrometer für quantitative Spektralmessungen geeignet ist, da die experimentell bestimmte Konstante nahezu deckungsgleich mit dem theoretischen Wert ist.

Die Analyse der Heliumlinien bestätigt die Zuverlässigkeit der Eichkurve. Die Abweichungen von maximal $1,7\sigma$ sind nicht signifikant und entsprechen den erwarteten Messungenauigkeiten. Dies deutet darauf hin, dass systematische Fehler, wie Abweichungen durch das Prisma oder die Kollimatorjustierung, nur eine untergeordnete Rolle spielen. Die genaue Fehlerquelle lässt sich nicht mit Sicherheit bestätigen.

Auffällig ist jedoch der Tend der Abweichungen aus \hyperref[tab:e_h]{Tabelle \ref*{tab:e_h}}, welche bei größeren Wellenlängen größere Abweichungen aufzeigen. Dies ist verwunderlich, da auf Grund der Kurvenstruktur, die Fehler für größere Wellenlängen größer werde, was die Abweichung verringert. Vermutlich liegt dies nicht an einem Messfehler des Heliumlampe, denn diese Sigmaabweichungen sind hintenhin statistisch signifikant zu den Literaturwerten. Viel mehr wird der Fehler an entweder einem Messfehler der Auslenkung bei der Wasserstofflampe liegen, oder an der Eichkurve, welche nachhintenhin ungenauer wurde. 

\section{Kritik}

Die Hauptquelle der Unsicherheit liegt im Ablesefehler des Nonius, der sich insbesondere bei schwach ausgeprägten Linien bemerkbar macht. Eine digitale Auswertung des Spektrums könnte diese Unsicherheiten reduzieren. Außerdem sind die Fehler durch die manuelle Erstellung der Eichkurve und das graphische Ablesen der Differenzen nur schwer präzise quantifizierbar. Für zukünftige Versuche wäre der Einsatz eines CCD-basierten Spektrometers oder einer automatisierten Winkelmessung empfehlenswert, um die Genauigkeit weiter zu erhöhen.  

Zusätzlich könnte die geringe Anzahl gemessener Linien die statistische Aussagekraft einschränken. Insbesondere die rote Linie der Wasserstofflampe zeigt eine deutlich größere Abweichung, die vermutlich durch Überlagerung von Linien oder geringe Intensität bedingt ist. Eine höhere Anzahl an Messungen pro Linie und die Aufnahme mehrerer Spektren könnte hier die Zuverlässigkeit verbessern.  

Insgesamt zeigt der Versuch jedoch, dass das verwendete Prismenspektrometer eine ausreichende Präzision für die Bestimmung diskreter Spektrallinien und abgeleiteter Größen wie der Rydberg-Konstante besitzt. Die Ergebnisse stehen im Einklang mit der Theorie, und die gemessenen Werte liegen überwiegend innerhalb der erwarteten Fehlergrenzen.
