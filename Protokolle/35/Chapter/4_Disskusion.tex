\chapter{Diskussion}

\section{Zusammenfassung}

Im Versuch wurde der photoelektrische Effekt untersucht, um das Plancksche Wirkungsquantum experimentell zu bestimmen.  
Hierzu wurde das Licht einer Quecksilberdampflampe mithilfe eines Prismenspektrometers in seine Spektrallinien zerlegt und jeweils auf die Kathode einer Fotozelle gerichtet.  
Für jede Linie wurde die Strom-Spannungs-Kennlinie aufgenommen, der Untergrundstrom subtrahiert und durch Extrapolation der lineare Bereich bestimmt, um die jeweilige Sperrspannung $U_S$ zu ermitteln.  
Die erhaltenen Sperrspannungen wurden anschließend gegen die Frequenzen der Spektrallinien aufgetragen.  
Aus der Steigung der linearen Ausgleichsgeraden wurde nach der Beziehung $h = e \, a$ das Plancksche Wirkungsquantum berechnet.  

Das experimentell bestimmte Ergebnis lautet:
\[
h = (7{,}7 \pm 0{,}6) \cdot 10^{-34}\,\text{Js}.
\]
Der Literaturwert beträgt:
\[
h_\text{lit} = 6{,}626070 \cdot 10^{-34}\,\text{Js}.
\]
Die Abweichung entspricht $1{,}79\,\sigma$ und liegt somit im Rahmen der Messunsicherheit.

\section{Diskussion}

Der ermittelte Wert des Planckschen Wirkungsquantums liegt etwa 16\,\% über dem Literaturwert, zeigt jedoch keine signifikante Abweichung, da die Unsicherheit des Messergebnisses vergleichsweise groß ist (ca. 5\%).  
Die Hauptquelle dieser Unsicherheit ist die grafische Bestimmung der Sperrspannungen.  
Da die Strom-Spannungs-Kennlinien von Hand ausgewertet wurden, führt insbesondere die subjektive Wahl des linearen Bereichs zu systematischen Abweichungen.  
Ebenso wirken sich Rundungsfehler und Ungenauigkeiten beim Ablesen der Messwerte auf die Regressionsanalyse aus.  

Ein weiterer Einflussfaktor ist die Ausrichtung des einfallenden Lichtstrahls auf die Fotozelle.  
Eine nicht zentrierte Beleuchtung kann den effektiven Photostrom verringern und damit die Bestimmung der Sperrspannung verfälschen.  
Zusätzlich besitzt das verwendete Multimeter eine begrenzte Genauigkeit, insbesondere im unteren Messbereich, was die Präzision der Spannungsmessung einschränkt.  
Die Annahme, dass der Fehler der Frequenzen vernachlässigbar ist, trägt ebenfalls zu einer leichten systematischen Verschiebung des Endergebnisses bei.

Trotz dieser Einschränkungen reproduziert das Ergebnis die lineare Abhängigkeit zwischen Sperrspannung und Lichtfrequenz, wie sie vom Einstein’schen Photoelektronengesetz vorhergesagt wird.  
Damit wird die quantisierte Energieübertragung des Lichts bestätigt und die Existenz von Photonen experimentell untermauert.

\section{Kritik}

Zur Verbesserung der Messgenauigkeit sollten die Auswertung und Bestimmung der Sperrspannungen numerisch erfolgen, z.\,B. durch lineare Regression auf digitalisierte Messdaten.  
Dies würde den Einfluss subjektiver Fehler bei der grafischen Auswertung eliminieren.  
Ferner könnten präzisere Messgeräte mit geringeren Messbereichsfehlern und automatischer Datenaufzeichnung eingesetzt werden.  
Eine bessere Justierung der optischen Komponenten, insbesondere der Strahlführung auf die Kathode, würde ebenfalls zur Verringerung systematischer Abweichungen beitragen.  

Insgesamt zeigt der Versuch trotz der genannten Ungenauigkeiten ein physikalisch konsistentes Ergebnis und bestätigt den quantenmechanischen Zusammenhang zwischen Lichtfrequenz und Elektronenenergie.
