\onecolumn
\chapter{Auswertung}
\label{ch:auswertung}

\section*{Fehlerrechnung}
Für die statistische Auswertung von $n$ Messwerten $x_i$ werden folgende Größen definiert \cite{errorSkript25}:
\begin{align}
    \bar{x} &= \frac{1}{n} \sum_{i=1}^{n} x_i \vphantom{\sqrt{\sum_i^n}^2} && \text{\textcolor{gray}{Arithmetisches Mittel}} \label{eq:arithmetisches_mittel} \\
    \sigma^2 &= \frac{1}{n-1} \sum_{i=1}^{n} (x_i - \bar{x})^2 \vphantom{\sqrt{\sum_i^n}^2} && \text{\textcolor{gray}{Variation}} \label{eq:variation} \\
    \sigma &= \sqrt{\frac{1}{n-1} \sum_{i=1}^{n} (x_i - \bar{x})^2} \vphantom{\sqrt{\sum_i^n}^2} && \text{\textcolor{gray}{Standardabweichung}} \label{eq:standardabweichung} \\
    \Delta \bar{x} &= \frac{\sigma}{\sqrt{n}} = \sqrt{\frac{1}{n(n-1)} \sum_{i=1}^n(\bar x - x_i)^2} \vphantom{\sqrt{\sum_i^n}^2} && \text{\textcolor{gray}{Fehler des Mittelwerts}} \label{eq:fehler_mittelwert} \\
    \Delta f &= \sqrt{\left(\frac{\partial f}{\partial x} \Delta x\right)^2 + \left(\frac{\partial f}{\partial y} \Delta y\right)^2} \vphantom{\sqrt{\sum_i^n}^2} && \text{\textcolor{gray}{Gauß'sches Fehlerfortpflanzungsgesetz für $f(x,y)$}} \label{eq:gauss_fehlfortpflanzung} \\
    \Delta f &= \sqrt{(\Delta x)^2 + (\Delta y)^2} \vphantom{\sqrt{\sum_i^n}^2} && \text{\textcolor{gray}{Fehler für $f = x + y$}} \label{eq:fehler_summe} \\
    \Delta f &= |a| \Delta x \vphantom{\sqrt{\sum_i^n}^2} && \text{\textcolor{gray}{Fehler für $f = ax$}} \label{eq:fehler_proportional} \\
    \frac{\Delta f}{|f|} &= \sqrt{\left(\frac{\Delta x}{x}\right)^2 + \left(\frac{\Delta y}{y}\right)^2} \vphantom{\sqrt{\sum_i^n}^2} && \text{\textcolor{gray}{relativer Fehler für $f = xy$ oder $f = x/y$}} \label{eq:relativer_fehler} \\
    \sigma &= \frac{|a_{lit} - a_{gem}|}{\sqrt{\Delta a_{lit}^2 + \Delta a_{gem}^2}} \vphantom{\sqrt{\sum_i^n}^2} && \text{\textcolor{gray}{Berechnung der signifikanten Abweichung}} \label{eq:signifikante_abweichung}
\end{align}

\twocolumn

\section{Bestimmung der Sperrspannung}
\label{sec:sperrspannung}

In diesem Abschnitt wird die Sperrspannung $U_S$ für alle vermessenen Spektrallinien bestimmt.
Zunächst werden die mit dem Multimeter gemessenen spannungswerte $U_I$ um den Dunkelstrom $U_{I0}$ korrigirt. Der korrigierte Spannungswert $U_k$ berechnet sich zu

\begin{equation}
\label{eq:uk}
U_k = U_I - U_{I0}.
\end{equation}

Der zugehörige Fehler ergibt sich nach dem Gauß'schen Fehlerfortpflanzungsgesetz unter Verwendung von \hyperref[eq:gauss_fehlfortpflanzung]{Gleichung \ref*{eq:gauss_fehlfortpflanzung}} zu

\begin{equation}
\label{eq:duk}
\Delta U_k = \sqrt{(\Delta U_I)^2 + (\Delta U_{I0})^2}.
\end{equation}

Die Einzelunsicherheiten $\Delta U_I$ und $\Delta U_{I0}$ setzen sich jeweils aus dem Ablesefehler und dem Multimeterfehler zusammen.
Für Messwerte größer als $4,\text{V}$ wurde der Fehler des Messbereichs $4,\text{V}$ verwendet ($\pm 0,4\% + 1$ digt.), für kleinere Spannungen der Fehler des Messbereichs $400,\text{mV}$ ($\pm 0,25\% + 5$ digt.).

Zur Bestimmung der Sperrspannung wird eine lineare Extrapolation des quadratischen Zusammenhangs zwischen Strom und Spannung durchgeführt. Dafür wird die Größe $U_p$ nach

\begin{equation}
\label{eq:up}
U_p = \sqrt{U_k}
\end{equation}

berechnet. Der zugehörige Fehler ergibt sich über partielle Ableitung aus \hyperref[eq:gauss_fehlfortpflanzung]{Gleichung \ref*{eq:gauss_fehlfortpflanzung}} zu

\begin{equation}
\label{eq:dup}
\Delta U_p = \frac{\Delta U_k}{2 \sqrt{U_k}}.
\end{equation}

Für jede Spektrallinie wird $U_p$ als Funktion der eingestellten Gegenspannung $U$ aufgetragen.
An den linearen Bereich der resultierenden Kennlinie wird eine Ausgleichsgerade angelegt. Die Sperrspannung $U_S$ wird als Schnittpunkt dieser Geraden mit der Abszisse bestimmt.

Der Fehler von $U$ wird vernachlässigt, da er im Vergleich zur Unsicherheit der Extrapolation gering ist. Der Fehler von $U_S$ ergibt sich überwiegend aus der Unsicherheit bei der Bestimmung des linearen Bereichs.
Zur graphischen Abschätzung wird daher eine Fehlergerade eingezeichnet, die den Grenzfall eines noch plausibel linearen Bereichs darstellt.

Die Ergebnisse der Berechnungen sind für jede Spektrallinie in einer Tabelle zusammengefasst.
Die resultierenden Sperrspannungen sind in Tabelle~\ref{tab:sperrspannungen} aufgeführt.

\begin{table}[h!]
\centering
\caption{Ermittelte Sperrspannungen der verschiedenen Spektrallinien.}
\label{tab:sperrspannungen}
\begin{tabular}{lccc}
\toprule
\textbf{Farbe} & $\boldsymbol{\lambda\,[\text{nm}]}$ & $\boldsymbol{f\,[\text{THz}]}$ & $\boldsymbol{U_S\,[\text{V}]}$ \\
\midrule
UV      & 365 & 821,3 & $-2{,}17 \pm 0{,}09$ \\
Violett & 405 & 740,2 & $-1{,}61 \pm 0{,}05$ \\
Blau    & 436 & 687,9 & $-1{,}418 \pm 0{,}015$ \\
Grün    & 546 & 549,0 & $-0{,}928 \pm 0{,}029$ \\
Gelb    & 578 & 518,7 & $-0{,}755 \pm 0{,}015$ \\
\bottomrule
\end{tabular}
\end{table}

\section{Bestimmung des Planckschen Wirkungsquantums}
\label{sec:planck}

In diesem Abschnitt wird das Plancksche Wirkungsquantum $h$ aus den zuvor bestimmten Sperrspannungen $U_S$ berechnet.  
Dazu werden die Sperrspannungen als Funktion der Frequenzen des eingestrahlten Lichts aufgetragen.  
Die Frequenzen werden aus den im Praktikumsskript angegebenen Wellenlängen übernommen, wobei kein zusätzlicher Fehler berücksichtigt wird.  
\hyperref[tab:sperrspannungen]{Tabelle \ref*{tab:sperrspannungen}} enthält die relevanten Werte.  

Aus der \hyperref[fig:final]{Abbildung \ref*{fig:final}} ergibt sich für den Betrag der Steigung der Ausgleichsgeraden:
\begin{align}
a &= 4,848 \cdot 10^{-15}\,\text{Vs} \label{eq:a_messwert}
\end{align}
Für die Fehlergerade wird erhalten:
\begin{align}
a' &= 4{,}428 \cdot 10^{-15}\,\text{Vs} \label{eq:a_fehlergerade}
\end{align}
Daraus folgt als gemitteltes Ergebnis:
\begin{align}
    \boxed{
        a = (4{,}8 \pm 0{,}4)\,\text{Vs} \label{eq:a}
    }
\end{align}

Das Plancksche Wirkungsquantum ergibt sich aus der Multiplikation der Steigung $a$ mit der Elementarladung $e$:
\begin{align}
h &= e \, a \label{eq:h}
\end{align}
mit
\begin{align}
e &= 1{,}602 \cdot 10^{-19}\,\text{C}. \label{eq:e}
\end{align}
Für die Fehlerfortpflanzung gilt gemäß \hyperref[eq:fehler_proportional]{Gleichung~\ref*{eq:fehler_proportional}}:
\begin{align}
\Delta h &= e \, \Delta a \label{eq:dh}
\end{align}

Eingesetzt ergibt sich:
\begin{align}
    \boxed{
        h = (7{,}7 \pm 0,6) \cdot 10^{-34}\,\text{Js} \label{eq:h_result}
    }
\end{align}

Zum Vergleich dient der Literaturwert
\begin{align}
h_\text{lit} &= 6{,}626070 \cdot 10^{-34}\,\text{Js}. \label{eq:h_lit}
\end{align}
Die Abweichung in Einheiten der Standardabweichung berechnet sich nach \hyperref[eq:signifikante_abweichung]{Gleichung~\ref*{eq:signifikante_abweichung}} zu:
\begin{align}
\frac{|h - h_\text{lit}|}{\Delta h} &= 1{,}79\,\sigma \label{eq:abweichung}
\end{align}
