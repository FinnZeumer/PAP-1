\chapter{Durchführung}

Für den Versuch wird ein klassisches Federpendel mit unbekannter Federkonstanten $D$ verwendet. Der Versuchsaufbau besteht aus den folgenden Hauptkomponenten:

\begin{itemize}
    \item \textbf{Feder:} Eine Zugfeder mit unbekannter Federkonstanten $D$, an der die Masse aufgehängt wird.
    \item \textbf{Massen:} Eine Auswahl an standardisierten Gewichten, die an die Feder gehängt werden können.
    \item \textbf{Halterung:} Eine stabile Halterung oder Stativ, an dem die Feder senkrecht aufgehängt wird.
    \item \textbf{Messinstrumente:} Ein Millimeterlineal zur Messung der Auslenkung inklusive zweier Markierungen, welche händisch gesetzt werden und eine händische Stoppuhr zur Messung der Schwingungsdauer. Diese Stoppuhr hat eine Präzision von $0,01\,\mathrm{s}$, aber unbekannter Genauigkeit \cite{Stoppuhr}.
\end{itemize}

Die Feder wird zunächst senkrecht an der Halterung befestigt. Unter das freie Ende der Feder wird die zu untersuchende Masse $m$ gehängt, sodass die Feder unter dem Einfluss der Gewichtskraft $mg$ ausgelenkt wird. Um die Schwingung zu starten, wird das Gewicht leicht aus der Gleichgewichtslage ausgelenkt, ohne dass eine zusätzliche Kraft auf das System wirkt.  

Die maximale Auslenkung sollte dabei klein gewählt werden, um die Kleinwinkeläherung der harmonischen Schwingung zu gewährleisten, jedoch groß genug, sodass die Auslenkung und schwingdauer präzieser Messbar sind.

\section{Messverfahren}
\subsection*{Maximaldurchgang vs. Nulldurchgang}

Zu Beginn wurde das Federpendel mit einer Masse von $200\,\mathrm{g}$ belastet. Anschließend wurde die Schwingungsdauer des Pendels mit zwei unterschiedlichen Methoden bestimmt.  

Für beide Methoden wurden jeweils zehn Messreihen aufgenommen, wobei in jeder Messreihe die Zeit für drei vollständige Schwingungen erfasst wurde und die Auslenkung etwa $100\pm 2 \, \mathrm{mm}$ betrug. 

\begin{itemize}
    \item \textbf{Methode 1:} Die Messung begann und endete jeweils am Punkt des maximalen Ausschlags des Pendels.  
    \item \textbf{Methode 2:} Die Messung begann und endete beim Durchgang durch die Ruhelage (Nulldurchgang).  
\end{itemize}

Für beide Methoden wurde aus den Messwerten die mittlere Schwingungsdauer bzw. die Periodendauer bestimmt. Zusätzlich wurde der mittlere Fehler des Mittelwertes berechnet, um die Genauigkeit der beiden Verfahren zu vergleichen. Die Methode mit der kleineren Unsicherheit wurde anschließend für die weiteren Messungen verwendet.  

\subsection*{Schwingungsdauer in Abhängigkeit von der Masse}
Im zweiten Teil des Versuchs wurde die Abhängigkeit der Schwingungsdauer $T$ von der angehängten Masse $m$ untersucht. Hierzu wurde das Pendel nacheinander mit Massen von $50\,\mathrm{g}$ bis $250\,\mathrm{g}$ in Schritten von $50\,\mathrm{g}$ belastet.  

Für jede Masse wurde die Zeit für drei Schwingungen dreimal gemessen. Aus den Messwerten wurde jeweils die mittlere Periodendauer bestimmt. Diese Messreihe diente im Anschluss zur Bestimmung der Federkonstante $D$ des Pendels.  

\subsection*{Statische Auslenkung in Abhängigkeit von der Masse}
Im letzten Versuchsteil wurde die statische Auslenkung des Federpendels in Abhängigkeit von der angehängten Masse ermittelt. Hierfür wurden nacheinander die Massen $0\,\mathrm{g}$\footnote{Eine Masse von 0g ergibt keinen Sinn, gemeint ist hier das Auslenken der Feder ohne angehängter Masse.}, $50\,\mathrm{g}$, $100\,\mathrm{g}$, $150\,\mathrm{g}$, $200\,\mathrm{g}$ und $250\,\mathrm{g}$ angebracht und die jeweilige Gleichgewichtsauslenkung $x_\text{stat}$ abgelesen.  

Die Ablesegenauigkeit des verwendeten Messinstruments wurde dokumentiert, um die spätere Fehleranalyse durchführen zu können. Diese Messung ermöglicht in Kombination mit der bekannten Federkonstanten $D$ die Bestimmung der Erdbeschleunigung $g$.  