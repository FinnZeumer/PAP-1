\chapter{Disskusion}

\section{Zusammenfassung}
Was haben wir gemacht? Das Ziel des Versuches ist es gewesen, eine akurate Bestimmung der Erdbeschleunigung $g$ über ein Federpendel vorzunehmen. 
Da die Federkonstante $D$ zunächst nicht bekannt war, musste auch diese Bestimmt werden, da diese essentiell zur Bestimmung der Erdbeschleunigung $g$ ist.
Auch vor der Bestimmung der Federkonstanten gab es etwas zu tun: Betsimmen, wie wir unsere Messung akkurat und präzise halten können, dafür wurden zwei verschiedene Messmethoden verglichen.

Zwar waren die Ergebnisse bei den beiden Methoden sehr vergleichbar, jedoch war der Nulldurchgang, der bessere mit einer standardabweichung von $0,030s$ (im Vergleich zu $0,043s$). 
Mit Dieser Messmethode haben wir dann die Federkonstante $D = (3,05 \pm 0,04) \frac{kg}{s^2}$ bestimmt. 
Diese wiederum steckte in der Gleichung für die Erdbeschleunigung, die Final auf einen Wert von $g_{gem} = 9,5 \pm 0,6 \frac{m}{s^2}$ berechnet wurde.
Dieser wert liegt im $0,49 \sigma$ bereich des Literaturwertes und ist somit signifikant.

\section{Disskusion}
In diesem versuch gibt es nicht besonders viel zu disskutieren. Was sich jedoch sagen lässt, ist das sowohl der statistische, als auch der systemische Fehler stark reduzieren lassen würde. Viele Fehlerquellen werden in der Kritik genannt.
Diese sind jedcoh alle systemische Fehler, und lassen sich nur durch besseres Equipment lösen. Aber der statistische Fehler ist davon unabhänig und ließe sich vor allem über größere Messreihen bestimmen. Für wirklich sinnvolle Ergebnisse 
hätte man hunderte bis tausende Messdaten jeweils haben müssen. Außerdem ist die Abschätzung des Fehlers sehr stark aus einem Bachgefühl entschieden wurden und auch dieser ließe sich sicherlich über genauere Methoden besser abschätzen.

\section{Kritik}
Bei der Durchführung und Auswertung des Versuchs zur Untersuchung des Hookeschen Gesetzes und der harmonischen Schwingung einer Feder sind zahlreiche potenzielle Fehlerquellen zu berücksichtigen. Diese lassen sich in unterschiedliche Kategorien einordnen und variieren stark in ihrer Relevanz für das Versuchsergebnis. Im Folgenden werden die wichtigsten Fehlerquellen systematisch aufgeführt, physikalisch eingeordnet und hinsichtlich ihrer Bedeutung bewertet.

\subsection*{1. Modellannahmen und Idealisierungen}

\begin{itemize}
    \item \textbf{Idealisierung des Hookeschen Gesetzes:} Das Hookesche Gesetz $F = -D \cdot s$ wurde im Versuch in seiner idealisierten linearen Form verwendet. Diese Annahme ist nur bei \emph{kleinen Auslenkungen} eine gültige Näherung. Bei größeren Auslenkungen wird das Kraft-Auslenkungs-Verhältnis zunehmend nichtlinear, wodurch das Modell an Genauigkeit verliert.
    
    \item \textbf{Nicht-ideale Eigenschaften realer Federn:} Reale Federn weisen Abweichungen vom idealen Modell auf, z.\,B.\ durch
    \begin{itemize}
        \item inhomogene Massenverteilung,
        \item interne Reibung (Hysterese),
        \item und eine nicht konstante Federkonstante.
    \end{itemize}
    Diese Effekte beeinflussen sowohl statische als auch dynamische Messungen.
\end{itemize}

\subsection*{2. Mess- und Geräteeinflüsse}

\begin{itemize}
    \item \textbf{Messgenauigkeit bei kleinen Auslenkungen:} Kleine Auslenkungen führen zu einem ungünstigen Verhältnis von Messfehler zur gemessenen Größe, insbesondere bei der Längenmessung, was die Bestimmung der Federkonstante beeinflussen kann.
    
    \item \textbf{Eichung und Kalibrierung der Geräte:} Alle verwendeten Messgeräte (Lineal, Stoppuhr, Waage) können systematische Fehler aufweisen. Eine unabhängige Kalibrierung war im Versuch nicht möglich. Auch die Massenstücke selbst können Fertigungstoleranzen besitzen.
    
    \item \textbf{Stativ und Befestigung:} Das Stativ war nicht vollständig starr. Dadurch konnten mechanische Schwingungen des Aufbaus das Schwingungsverhalten der Feder beeinflussen.
\end{itemize}

\subsection*{3. Einflüsse durch äußere Bedingungen}

\begin{itemize}
    \item \textbf{Reibung und Luftwiderstand:} Der Luftwiderstand ist bei kleinen Geschwindigkeiten gering, jedoch nicht vernachlässigbar. Er wirkt dämpfend auf die Schwingung und beeinflusst insbesondere die Periodendauer.
    
    \item \textbf{Geometrie der Massen:} Die verwendeten Massen waren nicht vollkommen symmetrisch. Dadurch kann der Luftwiderstand ungleichmäßig wirken und zu leichten Störungen führen.
    
    \item \textbf{Nicht-ideale Auslenkung:} Eine exakt vertikale Auslenkung der Feder war nicht immer gegeben. Dadurch entstanden Seitenschwingungen, was zusätzliche Reibungskräfte in mehreren Raumrichtungen verursachen kann.
\end{itemize}

\subsection*{4. Dynamikbedingte Näherungen}

\begin{itemize}
    \item \textbf{Vernachlässigung der Federmasse:} Die Eigenmasse der Feder wurde nicht berücksichtigt. Diese verändert die effektive Schwingungsmasse leicht und beeinflusst die berechnete Schwingungsdauer. Der Effekt war in diesem Versuch jedoch sehr gering.
    
    \item \textbf{Vernachlässigung der Eigenfrequenz der Feder:} Auch die Feder besitzt eine Eigenfrequenz. Diese wurde nicht berücksichtigt, hatte aber aufgrund der Parameterwahl nur einen vernachlässigbaren Einfluss.
    
    \item \textbf{Kopplung von Schwingungen:} Eine mögliche Kopplung von translatorischen und rotatorischen Schwingungen wurde im Modell nicht betrachtet. In der Versuchsdurchführung trat dieser Effekt nur geringfügig auf.
\end{itemize}

\subsection*{5. Menschliche Einflussfaktoren}

\begin{itemize}
    \item \textbf{Reaktionszeit bei der Zeitmessung:} Die größte Fehlerquelle stellt die menschliche Reaktionszeit beim Stoppen der Schwingungsdauer mit einer Handstoppuhr dar. Diese verursacht zufällige, aber nicht systematisch korrigierbare Fehler. Durch Mehrfachmessungen und Mittelwertbildung wurde versucht, diesen Effekt zu kompensieren, jedoch ist eine vollständige Eliminierung nicht möglich.
\end{itemize}

\subsection*{Fazit}

Zusammenfassend lässt sich feststellen, dass eine Vielzahl systematischer und zufälliger Fehlerquellen existieren. Viele davon – wie Luftwiderstand, Materialunvollkommenheiten oder Geräteungenauigkeiten – sind experimentell kaum vollständig vermeidbar. Besonders bedeutsam ist die Unsicherheit durch die manuelle Zeitmessung. Andere Effekte, wie die Vernachlässigung der Federmasse oder Modenkopplung, sind in diesem speziellen Versuchsaufbau hingegen vernachlässigbar. Eine präzisere Untersuchung wäre durch automatisierte Messsysteme, kalibrierte Messgeräte und eine differenziertere Modellierung realer Federverhalten möglich.
