\chapter{Auswertung}

\section{Aufgabe 1: Fehlerbestimmung zweier Messmethoden}
\subsection{Methode 1}
\begin{table}[h!]
    \centering
    \begin{tabular}{ccccc}\toprule
         Messung&  $T[s]$&  $\bar T[s]$& $\Delta T [s]$  & $\Delta \bar T [s]$\\\midrule
         1& 0,000 & 0,000 & 0,000 & 0,000\\
         2& 0,000 &  &  & \\
         3& 0,000 &  &  & \\
         4& 0,000 &  &  & \\
         5& 0,000 &  &  & \\
         6& 0,000 &  &  & \\
         7& 0,000 &  &  & \\
         8& 0,000 &  &  & \\
         9& 0,000 &  &  & \\
        10& 0,000 & & &\\ \bottomrule 
    \end{tabular}
    \caption{Periodendauer berechnet durch die Messung bei Maximalauslenkung des Pendels}
    \label{tab:periodendauer_Max}
\end{table}

\subsection{Methode 2}
\begin{table}[h!]
    \centering
    \begin{tabular}{ccccc}\toprule
         Messung&  $T[s]$&  $\bar T[s]$& $\Delta T [s]$  & $\Delta \bar T [s]$\\\midrule
         1& 0,000 & 0,000 & 0,000 & 0,000\\
         2& 0,000 &  &  & \\
         3& 0,000 &  &  & \\
         4& 0,000 &  &  & \\
         5& 0,000 &  &  & \\
         6& 0,000 &  &  & \\
         7& 0,000 &  &  & \\
         8& 0,000 &  &  & \\
         9& 0,000 &  &  & \\
        10& 0,000 & & &\\ \bottomrule 
    \end{tabular}
    \caption{Periodendauer berechnet durch die Messung bei Maximalauslenkung des Pendels}
    \label{tab:periodendauer_Max}
\end{table}

\section{Aufgabe 2: Messen der Schwingungsdauer als Funktion der Masse}
\section{Aufgabe 3: Bestimmung der Auslenkung als Funktion der Masse}
\section{Vergleich mit dem Literaturwert}