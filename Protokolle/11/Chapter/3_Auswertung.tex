\onecolumn
\chapter{Auswertung}
% Liste der genutzer Formeln für die Fehlerrechnung
\section*{Fehlerrechnung}
Für die statistische Auswertung von $n$ Messwerten $x_i$ werden folgende Größen definiert \cite{errorSkript25}:
\begin{align}
    \bar{x} &= \frac{1}{n} \sum_{i=1}^{n} x_i \vphantom{\sqrt{\sum_i^n}^2} && \text{\textcolor{gray}{Arithmetisches Mittel}}\\
    \sigma^2 &= \frac{1}{n-1} \sum_{i=1}^{n} (x_i - \bar{x})^2 \vphantom{\sqrt{\sum_i^n}^2} && \text{\textcolor{gray}{Variation}}\\
    \sigma &= \sqrt{\frac{1}{n-1} \sum_{i=1}^{n} (x_i - \bar{x})^2} \vphantom{\sqrt{\sum_i^n}^2} && \text{\textcolor{gray}{Standardabweichung}}\\
    \Delta \bar{x} &= \frac{\sigma}{\sqrt{n}} = \sqrt{\frac{1}{n(n-1)} \sum_{i=1}^n(\bar x - x_i)^2} \vphantom{\sqrt{\sum_i^n}^2} && \text{\textcolor{gray}{Fehler des Mittelwerts}}\\
    \Delta f &= \sqrt{\left(\frac{\partial f}{\partial x} \Delta x\right)^2 + \left(\frac{\partial f}{\partial y} \Delta y\right)^2} \vphantom{\sqrt{\sum_i^n}^2} && \text{\textcolor{gray}{Gauß’sches Fehlerfortpflanzungsgesetz für $f(x,y)$}}\\
    \Delta f &= \sqrt{(\Delta x)^2 + (\Delta y)^2} \vphantom{\sqrt{\sum_i^n}^2} && \text{\textcolor{gray}{Fehler für $f = x + y$}}\\
    \Delta f &= |a| \Delta x \vphantom{\sqrt{\sum_i^n}^2} && \text{\textcolor{gray}{Fehler für $f = ax$}}\\
    \frac{\Delta f}{|f|} &= \sqrt{\left(\frac{\Delta x}{x}\right)^2 + \left(\frac{\Delta y}{y}\right)^2} \vphantom{\sqrt{\sum_i^n}^2} && \text{\textcolor{gray}{relativer Fehler für $f = xy$ oder $f = x/y$}}\\
    \sigma &= \frac{|a_{lit} - a_{gem}|}{\sqrt{\Delta a_{lit}^2 + \Delta a_{gem}^2}} \vphantom{\sqrt{\sum_i^n}^2} && \text{\textcolor{gray}{Berechnung der signifikanten Abweichung}}
\end{align}

\twocolumn

\section{Fehlerbestimmung zweier Messmethoden}
\subsection*{Methode 1}
\begin{table}[h!]
    \centering
    \begin{tabular}{ccccc}\toprule
         Messung&  $T[s]$&  $\bar T[s]$& $\Delta T [s]$  & $\Delta \bar T [s]$\\\midrule
         1& 0,000 & 0,000 & 0,000 & 0,000\\
         2& 0,000 &  &  & \\
         3& 0,000 &  &  & \\
         4& 0,000 &  &  & \\
         5& 0,000 &  &  & \\
         6& 0,000 &  &  & \\
         7& 0,000 &  &  & \\
         8& 0,000 &  &  & \\
         9& 0,000 &  &  & \\
        10& 0,000 & & &\\ \bottomrule 
    \end{tabular}
    \caption{Periodendauer berechnet durch die Messung bei Maximalauslenkung des Pendels}
    \label{tab:periodendauer_Max}
\end{table}

\subsection*{Methode 2}
\begin{table}[h!]
    \centering
    \begin{tabular}{ccccc}\toprule
         Messung&  $T[s]$&  $\bar T[s]$& $\Delta T [s]$  & $\Delta \bar T [s]$\\\midrule
         1& 0,000 & 0,000 & 0,000 & 0,000\\
         2& 0,000 &  &  & \\
         3& 0,000 &  &  & \\
         4& 0,000 &  &  & \\
         5& 0,000 &  &  & \\
         6& 0,000 &  &  & \\
         7& 0,000 &  &  & \\
         8& 0,000 &  &  & \\
         9& 0,000 &  &  & \\
        10& 0,000 & & &\\ \bottomrule 
    \end{tabular}
    \caption{Periodendauer berechnet durch die Messung bei Maximalauslenkung des Pendels}
    \label{tab:periodendauer_Max}
\end{table}

\section{Messen der Schwingungsdauer als Funktion der Masse}
\section{Bestimmung der Auslenkung als Funktion der Masse}
\section{Vergleich mit dem Literaturwert}