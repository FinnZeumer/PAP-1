\onecolumn
\chapter{Evaluation}
% List of formulas used for error analysis
\section*{Error Analysis}
For the statistical evaluation of $n$ measured values $x_i$, the following quantities are defined \cite{errorSkript25}:
\begin{align}
    \bar{x} &= \frac{1}{n} \sum_{i=1}^{n} x_i \vphantom{\sqrt{\sum_i^n}^2} && \text{\textcolor{gray}{Arithmetic mean}} \label{eq:arithmetisches_mittel} \\
    \sigma^2 &= \frac{1}{n-1} \sum_{i=1}^{n} (x_i - \bar{x})^2 \vphantom{\sqrt{\sum_i^n}^2} && \text{\textcolor{gray}{Variance}} \label{eq:variation} \\
    \sigma &= \sqrt{\frac{1}{n-1} \sum_{i=1}^{n} (x_i - \bar{x})^2} \vphantom{\sqrt{\sum_i^n}^2} && \text{\textcolor{gray}{Standard deviation}} \label{eq:standardabweichung} \\
    \Delta \bar{x} &= \frac{\sigma}{\sqrt{n}} = \sqrt{\frac{1}{n(n-1)} \sum_{i=1}^n(\bar x - x_i)^2} \vphantom{\sqrt{\sum_i^n}^2} && \text{\textcolor{gray}{Error of the mean}} \label{eq:fehler_mittelwert} \\
    \Delta f &= \sqrt{\left(\frac{\partial f}{\partial x} \Delta x\right)^2 + \left(\frac{\partial f}{\partial y} \Delta y\right)^2} \vphantom{\sqrt{\sum_i^n}^2} && \text{\textcolor{gray}{Gaussian error propagation for $f(x,y)$}} \label{eq:gauss_fehlfortpflanzung} \\
    \Delta f &= \sqrt{(\Delta x)^2 + (\Delta y)^2} \vphantom{\sqrt{\sum_i^n}^2} && \text{\textcolor{gray}{Error for $f = x + y$}} \label{eq:fehler_summe} \\
    \Delta f &= |a| \Delta x \vphantom{\sqrt{\sum_i^n}^2} && \text{\textcolor{gray}{Error for $f = ax$}} \label{eq:fehler_proportional} \\
    \frac{\Delta f}{|f|} &= \sqrt{\left(\frac{\Delta x}{x}\right)^2 + \left(\frac{\Delta y}{y}\right)^2} \vphantom{\sqrt{\sum_i^n}^2} && \text{\textcolor{gray}{Relative error for $f = xy$ or $f = x/y$}} \label{eq:relativer_fehler} \\
    \sigma &= \frac{|a_\text{lit} - a_\text{meas}|}{\sqrt{\Delta a_\text{lit}^2 + \Delta a_\text{meas}^2}} \vphantom{\sqrt{\sum_i^n}^2} && \text{\textcolor{gray}{Calculation of significant deviation}} \label{eq:signifikante_abweichung}
\end{align}

\twocolumn

\section{Pendulum Oscillation Period}
First, the period of an undamped harmonic oscillator is determined. The measured data are shown in \hyperref[tab:oscillation_data]{Table 1}:

\begin{table}[h!]
    \centering
    \begin{tabular}{c|c|c|c|c}
        \hline
        \textbf{Index} & \textbf{$t$ [s]} & \textbf{$\Delta t$ [s]} & \textbf{$T$ [s]} & \textbf{$\Delta T$ [s]} \\
        \hline
        1 & 35.37 & 0.20 & 1.7685 & 0.01 \\
        2 & 35.53 & 0.20 & 1.7765 & 0.01 \\
        3 & 35.29 & 0.20 & 1.7645  & 0.01 \\
        \hline
    \end{tabular}
    \caption{Measured oscillation times for 20 oscillations and calculated periods. Uncertainties in time and period are indicated by $\Delta t$ and $\Delta T$, respectively.}
    \label{tab:oscillation_data}
\end{table}

The time uncertainty $\Delta t$ is based on an estimated human reaction time of 200 ms. The stopwatch error is negligible. 

From \hyperref[tab:oscillation_data]{Table 1}, the average period is calculated as
\begin{equation}
    \boxed{\bar{T_0} = (1.77 \pm 0.01)\,\mathrm{s}}
\end{equation}

\section{Calculation of the Damping Constant}
With the period $T_0$ known, the next step is to determine the damping constant $\delta$ using graphical analysis. Data from the protocol (\hyperref[tab:settling_times]{Tables 2 and 3}) are used to construct a decay curve.

\begin{table}[h!]
    \centering
    \begin{tabular}{c|c|c}
        \hline
        \textbf{$I$ [mA]} & \textbf{Periods} & \textbf{$t$ [s]} \\
        \hline
        $340 \pm 10$ & 15 & $26.81 \pm 0.20$ \\
        $440 \pm 10$ & 10 & $17.42 \pm 0.20$ \\
        \hline
    \end{tabular}
    \caption{Measured settling times and number of periods for two different current values. Uncertainties are indicated for current and time.}
    \label{tab:settling_times}
\end{table}

The corresponding periods for the two damping currents are calculated as:
\begin{equation}
    \boxed{
        \begin{aligned}
        T_{340\,\mathrm{mA}} &= (1.587 \pm 0.013)\,\mathrm{s} \\
        T_{440\,\mathrm{mA}} &= (1.742 \pm 0.020)\,\mathrm{s}
    \end{aligned}
    }
\end{equation}


We get the results of \hyperref[protocol]{Tabel 3 from the protocol} and seperate it into two seperate: 
\begin{table}[h!]
    \centering
    \begin{tabular}{c|ccc|c}
        \hline
        \textbf{Index} & \textbf{$A_1$ [Un]} & \textbf{$A_2$ [Un]} & \textbf{$A_{Avg}$ [Un]} & \boldmath$\Delta$ \textbf{$A_{Avg}$} \\
        \hline
        1  & 15.0 & 14.6 & 14.8 & 0.4 \\
        2  & 13.0 & 12.2 & 12.6 & 0.8 \\
        3  & 11.0 & 10.2 & 10.6 & 0.8 \\
        4  & 8.8  & 8.6  & 8.7  & 0.2 \\
        5  & 7.6  & 6.8  & 7.2  & 0.8 \\
        6  & 6.4  & 5.8  & 6.1  & 0.6 \\
        7  & 5.2  & 4.6  & 4.9  & 0.6 \\
        8  & 4.6  & 3.8  & 4.2  & 0.8 \\
        9  & 3.8  & 3.0  & 3.4  & 0.8 \\
        10 & 3.2  & 2.4  & 2.8  & 0.8 \\
        11 & 2.6  & 1.8  & 2.2  & 0.8 \\
        12 & 2.2  & 1.4  & 1.8  & 0.8 \\
        13 & 1.8  & 1.0  & 1.4  & 0.8 \\
        14 & 1.6  & 0.8  & 1.2  & 0.8 \\
        15 & 1.2  & 0.4  & 0.8  & 0.8 \\
        \hline
    \end{tabular}
    \caption{Measured amplitudes for 340mA with Units and the uncertainty.}
    \label{tab:340mA}
\end{table}

\begin{table}[h!]
    \centering
    \begin{tabular}{c|ccc|c}
        \hline
        \textbf{Index} & \textbf{$A_1$ [Un]} & \textbf{$A_2$ [Un]} & \textbf{$A_{Avg}$ [Un]} & \boldmath$\Delta$ \textbf{$A_{Avg}$} \\
        \hline
        1  & 14.2 & 13.2 & 13.7 & 1.0 \\
        2  & 10.4 & 10.0 & 10.2 & 0.4 \\
        3  & 8.0  & 7.4  & 7.7  & 0.6 \\
        4  & 5.8  & 5.2  & 5.5  & 0.6 \\
        5  & 4.2  & 3.8  & 4.0  & 0.4 \\
        6  & 3.6  & 2.6  & 3.1  & 1.0 \\
        7  & 2.6  & 1.8  & 2.2  & 0.8 \\
        8  & 2.0  & 1.4  & 1.7  & 0.6 \\
        9  & 1.0  & 0.8  & 0.9  & 0.2 \\
        10 & 1.0  & 0.4  & 0.7  & 0.6 \\
        \hline
    \end{tabular}
    \caption{Measured amplitudes for 440mA with Units and the uncertainty.}
    \label{tab:440mA}
\end{table}



\begin{table}[h!]
    \centering
    \begin{tabular}{l|l}
        \hline
        \textbf{Frequency [Hz]} & \textbf{Amplitude [Units]} \\
        \hline
        $301 \pm 0.1$ & 0.6 \\
        501           & 0.6 \\
        701           & 0.6 \\
        901           & 0.7 \\
        1100          & 0.7 \\
        1300          & 0.8 \\
        1500          & 0.9 \\
        1700          & 1.0 \\
        1898          & 1.3 \\
        \rowcolor{blue!20} \textcolor{gray}{Switch >>1K<<} & to \textcolor{gray}{>>10K<<}\\
        $2112 \pm 4$  & 2.3 \\
        2304          & 4.8 \\
        2524          & 1.6 \\
        2700          & 1.1 \\
        2908          & 0.8 \\
        3100          & 0.7 \\
        3307          & 0.6 \\
        3517          & 0.5 \\
        3797          & 0.5 \\
        3912          & 0.4 \\
        \hline
        \hline
        \rowcolor{red!20} \textcolor{gray}{Find actual} & \textcolor{gray}{max amplitude}\\
        \textcolor{red}{ADD MISSING VALUES}
    \end{tabular}
    \caption{Amplitude measurements at 340 mA across increasing frequencies. The uncertainty at 301 Hz is $\pm 0.1$ Vpp, and at 2112 Hz is $\pm 4$ Hz. A mode switch in the device occurred at 2112 Hz, indicated by the blue line in the original data.}
    \label{tab:340mA_mode_switch}
\end{table}


\label{tab:yes_i_nee_to_add_english_labels_now_well_actually_i_could_still_make_them_german_but_this_feels_illigal_HELP_ME_if_you_will_ever_find_this_HELP_ME_I_am_going_crazy_hahahahahahahaha_just_kidding}