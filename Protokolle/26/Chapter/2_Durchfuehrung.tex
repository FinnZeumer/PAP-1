\chapter{Durchführung}

Im ersten Teil des Experiments wurde die Schallgeschwindigkeit mit einem Quincke'schen Rohr bestimmt. Der Versuchsaufbau bestand aus einem vertikal montierten Rohr, das am oberen Ende mit einem Lautsprecher verbunden war und am unteren Ende eine verstellbare Wassersäule enthielt. Über einen Sinusgenerator wurde der Lautsprecher mit einer sinusförmigen Wechselspannung gespeist. Das Ausgangssignal des Generators wurde zusätzlich über einen digitalen Lock-In-Verstärker geführt, dessen Ausgangsspannung auf einem analogen Multimeter im 10-V-Bereich abgelesen wurde. Die Wassersäule diente als reflektierende Grenzfläche, deren Höhe über ein Ausgleichsgefäß präzise eingestellt werden konnte. Für Messungen in Kohlendioxid wurde das Rohr zusätzlich über eine Gasflasche mit Reduzierventil, Drucktastenventil und Zuleitungsschlauch mit CO$_2$ gefüllt. Ein brennendes Streichholz diente zur Überprüfung, ob das Gas die Luft vollständig verdrängt hatte.

Im zweiten Versuchsteil wurde die Schallgeschwindigkeit über die direkte Laufzeitmessung einer fortschreitenden Schallwelle bestimmt. Hierzu befanden sich ein Lautsprecher und ein verschiebbares Mikrofon in einem schallgedämmten Kasten. Das vom Sinusgenerator erzeugte Signal wurde gleichzeitig an den Lautsprecher und an Kanal~1 des Oszilloskops aneglegt. Das Mikrofon wandelte die ankommende Schallwelle wieder in ein elektrisches Signal um, das an Kanal~2 des Oszilloskops weitergegeben wurde. Durch Vergleich der beiden Signale konnte die Phasenverschiebung und damit die Laufzeit bestimmt werden. Eine schematische Darstellung beider Aufbauten befindet sich im \hyperref[Protokoll]{Protokoll} .

\subsection*{Aufgabe 1: Messung mit dem Quincke'schen Rohr}

Zu Beginn der Messung wurde das Rohr mit Luft gefüllt. Die Frequenz des Sinus Generators wurde auf einen Wert zwischen 2 und 3\,kHz eingestellt und die Ausgangsspannung so gewählt, dass das multimeter etwa 8\,V anzeigte. Werte oberhalb von 12\,V wurden vermieden, um Übersteuerungen des Digital-Analog-Wandlers zu verhindern. Da der Lock-In-Verstärker sehr empfindlich reagierte, war der Ton nur schwach hörbar. Entscheidend war das Spannungsmaximum am Multimeter, das auf Resonanzbedingungen hinwies.

Zur Ermittlung der Wellenlänge wurde der Wasserstand im Rohr langsam verändert, bis ein Lautstärkemaximum erreicht wurde. Die Höhe der Wassersäule wurde notiert und der Vorgang mehrfach wiederholt, um die Abstände $h_i$ zwischen aufeinanderfolgenden Resonanzhöhen zu bestimen. Aus den Differenzen der Höhen ließ sich die Wellenlänge $\lambda$ nach \hyperref[eq:wellenlaenge]{Gleichung~\ref*{eq:wellenlaenge}} berehcnen. Die Frequenz wurde vor und nach jeder Messreihe überprüft, um eventuelle Schwankungen zu erfassen. Anschließend wurde die Messung mit CO$_2$ wiederholt. Dazu wurde das Rohr durch vorsichtiges Einleiten von CO$_2$ von unten nach oben befüllt, bis das Gas die Luft vollständig verdrängt hatte. Ein erlöschendes Streichholz an der Öffnung diente als Kontrolle. Der Lautsprecher wurde wieder eingesetzt, und die Resonanzmessung erfolgte analog zur Luftmessung. Nach jeder Absenkung des Wasserstandes wurde gegebenenfalls weiteres CO$_2$ zugeführt. Am Ende der Messung wurden Temperatur und Gasdruck notiert, um die Werte später auf Normalbedingungen umzurechnen. Danach wurde das Hauptventil der Gasflasche geschlossen und das Wasser vollständig abgelassen.

\subsection*{Aufgabe 2: Messung der Laufzeit einer fortschreitenden Welle}

Im zweiten Versuchsteil wurde die Schall Geshwindigkeit durch direkte Laufzeitmessung bestimmt. Der Sinusgenerator wurde auf eine frequenz von 10\,kHz eingestellt. Das Ausgangssignal wurde gleichzeitig an den Lautsprehcer und an Kanal~1 des Oszilloskops gelegt, während das Mikrofonsignal an Kanal~2 angeschlossen wurde. Beim Einschalten des Signals erschienen auf dem Oszilloskopschirm zwei sinusförmige Spannungen, die eine Phasenverschiebung aufwiesen. Durch Verschieben des Mikrofons entlang der Ausbreitungsrichtung der Schallwelle änderte sich der Abstand $h$ zwischen Lautsprecher und Mikrofon, was eine Veränderung der Phasenlage bewirkte. Die Laufzeit $\tau$ konnte über die Phasenverschiebung gemäß \hyperref[eq:laufzeit]{Gleichung~\ref*{eq:laufzeit}} bestimmt werden.

Für die Auswertung wurden alle Positionen ermittelt, bei denen sich das Signal auf dem Oszilloskop um genau eine Periode, also um eine Phasenverschiebung von 360°, verschoben hatte. Die Messung wurde zweimal wiederholt. Aus den Differenzen der gemessenen Abstände ergab sich die Wellenlänge $\lambda$, und mithilfe von \hyperref[eq:c_aus_lambda]{Gleichung~\ref*{eq:c_aus_lambda}} konnte die Schallgeschwindigkeit berechnet werden. Zur Kontrolle wurde die Periodendauer $T$ der Generatorfrequenz am Oszilloskop gemessen, woraus eine Frequenz von 10\,kHz bestimmt wurde.

\subsection*{Aufgabe 3: Zusatzmessungen}

Zum Abschluss wurde das Frequenzspectrum der menschlichen Stimme betrachtet. Hierzu wurde der Deckel des schallgedämmten Kastens geöffnet, und die Stimme wurde in Richtung des Mikrofons gesprochen. Auf dem Oszilloskop konnten die Frequenzanteile qualitativ beobachtet werden. Zusätzlich wurde überprüft, ob die Schallgeschwindigkeit frequenzunabhängig ist, indem Messungen bei 2\,kHz und 5\,kHz durchgeführt wurden. Dazu wurden für jede Frequenz zwei Mikrofonpositionen mit bekanntem Phasenunterschied bestimmt. Die Ergebnisse bestätigten, dass die Schallgeschwindigkeit unabhängig von der Anregungsfrequenz ist.

Die gemessenen Daten dienten als Grundlage für die anschließende Auswertung. Zur Umrechnung auf Normalbedingungen wurde \hyperref[eq:normbedingungen]{Gleichung~\ref*{eq:normbedingungen}} verwendet, während theoretische Vergleichswerte über \hyperref[eq:theorie_c]{Gleichung~\ref*{eq:theorie_c}} bestimmt wurden.
